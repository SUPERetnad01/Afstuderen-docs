\chapter{Kwaliteitsbewaking}
In dit hoofdstuk wordt beschreven hoe de kwaliteit van het product gewaarborgd wordt.
Dit hoofdstuk is onderverdeeld in: Scrum of One, Code reviews, Versiebeheer, ontwerpen, Testing en Documentatie.
\section{Scrum of One}
De ontwerp en realisatie fases van het project worden ondersteund door middel van de scrum methodiek.
Omdat de opdracht alleen wordt uitgevoerd wordt er een variatie gebruikt de Scrum of one methodiek \Parencite{ScrumOfOne}.
\whitespace
Tijdens de de realisatieperiode wordt de SDLC-cyclus doorlopen.
Tijdens de requirement analysis wordt behandeld in het onderzoek. 
Daarna worden er Y sprint geplant voor het designen van het systeem.
Vervolgens worden er X-Y sprints geplant voor het implementeren en testen van het systeem. \\
\textbf{requirement analysis:} door middel van de requirement analyse die wordt uitgevoerd worden user stories gerefined tot kleine duidelijke eisen en wensen, zodat ze mee kunnen genomen in de sprint planning.\\
\textbf{Design:} De eisen en wensen vanuit de de requirement analyse worden gebruikt om het ontwerp te maken.
De ontwerpen kunnen ook uitgebreid worden of extra ontwerpen gemaakt.
\textbf{Inplementation:}
De eisen en wensen worden geimplementeerd volgens het ontwerp binnen het systeem.
\textbf{Testing:} De implementatie wordt getest doormidel van het V-Model, zie hoofdstuk \ref{Testing} voor meer informatie.\\
\textbf{Evolution:} Er wrodt
\todo[inline]{is 1 sprint te weinig en zijn 2 sprints te veel}
De X-1 sprints 
\todo[inline]{check hoeveel sprints er zijn voor het realiseren/ontwerpen van de applicatie}
\section{Code reviews}
er worden code reviews gehouden om de code kwaliteit te kunnen waarborgen.
\section{Versiebeheer}
Er wordt gewerkt met git en bitbucket, hier wordt de code opgeslagen. Verder wordt er een CI pipeline opgezet voor het automatisch testen van de code.
\section{Ontwerpen}
dit is ontwerpen wow
\section{Testing}
\label{sec:Testing}
V model
\section{Documentatie}
praten met bedrijfsbegeleider en afstudeerbegeleider.
% \section{Software testing}
% \section{Continuous Integration}
% \section{User testing}
