\section{Probleemstelling}
% Het huidige Snakware platorm is een 21 jaar oud platform, waar nog steeds aangewerkt wordt en functionaliteit aan toegevoegt wordt.
% Dit heeft er voor gezorgd dat de datastructuur en complexiteit van het systeem met de jaren ook verhoogd. 
% Vanwege deze complexiteit kost het nu veel tijd en geld om nieuwe functionaliteit in het huidige platform te bouwen.
% Daarnaast zijn de veel technieken en best practices gebruikt die nu niet meer als optimaal worden beschouwed.
% \whitespace
\todo[inline]{Stukje voor het beschrijven dat het nu niet mogelijk is om kleinere klanten te onboarden.}
Het huidige platform is 21 jaar oud en er is veel functionaliteit in de loop der jaren aan toegevoegt.
Omdat Snakeware Cloud een oud platform is zijn er veel technieken en best practices gebruikt die nu niet meer als optimaal worden beschouwed.
Een voorbeeld hier van is tabel naam afkortingen voor elke kolom zetten in de database, of hele grote code files met alle functionaliteiten daar in van meerdere duizenden regels lang. 
Ook zijn er technieken toegepast die nu niet meer relavant zijn, een voorbeeld hiervan is dat het \gls{CMS} gebruik maakt van javascript en toen ze er mee begonnen bestonden javascript classes nog niet dus hebben ze die zelf geimplenteerd.
Er zijn veel van dit soort gevallen wat het lastig maakt om Snakeware cloud te onderhouden of uit te breiden.
\whitespace
Een van de voornaamste kwesties met Snakeware Cloud betreft de verouderde datastructuur van de applicatie.
Deze veroudering is het gevolg van een initiële ontwikkeling waarbij onvoldoende rekening werd gehouden met toekomstige functionaliteitsuitbreidingen in het systeem.
Als gevolg daarvan is de onderliggende datastructuur niet aangepast, maar zijn er elementen aan toegevoegd.
Dit heeft geresulteerd in database queries van duizenden regels en complexe relaties tussen tabellen in de database.
Dit huidige scenario bemoeilijkt aanzienlijk het toevoegen van nieuwe functionaliteiten, wat resulteert in aanzienlijke tijds- en kosteninvesteringen.
% \whitespace
% Hierom wilt Snakeware dat er een nieuwe datastructuur komt met daar bij een applicatie.
% Omdat er een nieuwe datastructuur moet komen en de logica van het oude systeem nauw verbonden is met de
% datastructuur is het niet mogelijk om de oude code opnieuw te kunnen gebruiken.
% Dit platform moet een grote hoeveelheid potentiële kleine klanten kunnen ondersteunen naast de grotere klanten van Snakeware.
% \todo[inline]{Probleemstelling op projectniveau}
% \todo[inline]{https://www.scribbr.nl/starten-met-je-scriptie/verschil-tussen-probleemstelling-hoofdvraag-onderzoeksvraag-en-doelstelling/}
