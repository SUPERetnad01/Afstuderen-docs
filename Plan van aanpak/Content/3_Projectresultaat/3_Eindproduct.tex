\section{Eindproduct}
Om de gestelde doelen te bereiken, zullen er vier eindproducten worden ontwikkeld.
Deze eindresultaten omvatten het projectresultaat, dat aan het einde van de afstudeerperiode wordt gepresenteerd en gedemonstreerd.
De vier producten in kwestie zijn: het Plan van Aanpak, het Onderzoeksverslag, het Technisch Verslag en het Eindproduct.
\whitespace
\textbf{Plan van Aanpak}: Dit document beschrijft in detail de uitvoering van de opdracht.
\whitespace
\textbf{Onderzoeksverslag}: In het onderzoeksverslag staat het uigewerkde ondezoek.
Het onderzoek zal een lijst van requirements opleveren die gebruikt worden tijdens de ontwerp fase en realisatie fase.
\whitespace
\textbf{Technisch verslag}: In het Technisch Verslag worden de beslissingen die zijn genomen tijdens het uitvoeringsproces duidelijk uiteengezet, en wordt het ontwerp gepresenteerd met onderbouwing van de gemaakte keuzes.
Voor het opstellen van zowel het ontwerp als de implementatie is het van cruciaal belang om rekening te houden met de eisen en wensen die zijn voortgekomen uit de requirementanalyse. \\
\textbf{Ontwerpen (Ontwerpdocument)}: In dit stadium worden ontwerpen gemaakt door middel van de uitwerking van een 4 + 1 model, waarin de architectuur van het systeem duidelijk zichtbaar wordt.\\
\textbf{Realiseren (Implementatiedocument)}: Het Technisch Verslag voorziet in een gedetailleerde beschrijving van hoe het systeem functioneert, inclusief de keuzes die zijn gemaakt op basis van het ontwerp. \\
\textbf{Test (Validatie- en Verificatieplan)}: Het systeem wordt onderworpen aan verificatie en validatie volgens het V-Model.                                                                                             \\
\textbf{Reflectie (Reflectieverslag)}: Het laatste deel van het Technisch Verslag omvat een reflectie op het realisatie- en ontwerpproces, waarbij de STARR-methode \Parencite{STARR} wordt toegepast voor een grondige evaluatie.
\whitespace
\textbf{Product}: Het product omvat de uitwerking van het proof of concept. zie meer informatie hoofdstuk 1.4 en het opleveren van de broncode naar Snakeware.
