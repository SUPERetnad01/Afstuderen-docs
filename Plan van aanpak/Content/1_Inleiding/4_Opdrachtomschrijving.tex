\section{Opdrachtomschrijving}
\label{sec:Opdrachtomschrijving}
De opdracht is om een proof of concept CMS-API te ontwikkelen die gebruikt maakt van een datamodel en systeemarchitectuur dat flexibeler, onderhoudbaarder en gebruikt maakt van moderne best practices.
Tijdens de afstudeeropdracht wordt er primair op het datamodel en de systeemarchitectuur gefocust.
Omdat er nog geen concreet datamodel en systeemarchitectuur is zal dit onderzocht/ontworpen moeten worden.

\whitespace[2]
De opdracht omvat het achterhalen van de requirements, ontwerpen en ontwikkelen van het proof of concept met als focus een nieuw datamodel, met de essentiële functionaliteiten.
% Omdat dit een proof of concept project is, wordt er gebruikgemaakt van gekwalificeerde interne stakeholders die de wensen van de klanten kunnen vertegenwoordigen om hier de functionele requirements uit op te halen.
% \todo[inline]{\textit{Wat maakt de stakeholder gekwalificeerd?} was een van de opmerkingen van stefan. Heb de neiging om de bovenste zin weg te halen omdat we in theorie niet weten wie de stakeholders zijn omdat dat deelvraag 1 is.}

\whitespace[2]
Het huidige Snakeware cloud platform bestaat uit 2 verschillende \gls{GUI}:
\begin{itemize}
	\item[-] Snakeware cloud \gls{GUI}
	\item[-] klant webapplicatie
\end{itemize}

\whitespace
Met de Snakeware cloud \gls{GUI} kan de klant de content van de website aanpassen.
Door middel van de webapplicatie kan de eindgebruiker de content bekijken en interacteren.
Er is voor gekozen om niet de Snakeware cloud \gls{GUI} te realiseren om de afstudeeropdracht in scope te houden.
Er is wel voor gekozen om de klant webapplicatie in zijn minimaal uit te werken.

\whitespace[2]
Om de \gls{UserJourneys} te testen wordt er gebruikgemaakt van postman workflows \Parencite{PostmanWorkflows}.
Het doel van het proof of concept is dat er aangetoond kan worden dat door het gebruiken van een nieuw datamodel en systeemarchitectuur ook services verleend kunnen worden aan kleinere klanten.
Dit zou eventueel ook een startpunt zijn om op verder te bouwen.
