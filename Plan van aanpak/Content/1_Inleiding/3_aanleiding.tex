\section{Aanleiding}
Het huidige platform is 21 jaar oud en er is veel functionaliteit in de loop der jaren aan vast geplakt. Dit
heeft ervoor gezorgd dat de onderliggende datastructuur erg complex is geworden.
Het kost nu erg veel tijd en geld on nieuwe functionaliteiten in het CMS te maken omdat het zo erge complexe datastructuur heeft.
In de loop van het ontwikkel process zijn tijdens de ontwikkelingen van de features (de toen nog niet bestaande) best practices toegepast, 
\todo[inline]{niet duidelijk wat hiermee bedoelt word. examen commisie: Het lijkt on intressant om dat (de uitleg dus ook) terug te zien in de uitvoering}{hierdoor is de logica sterk gekoppeld met de database.(hier moet misschien meer over uitgeled worden in een ander hoofdstuk)}
\whitespace
Hierom wilt Snakeware dat er een nieuwe datastructuur komt met daar bij een nieuwe applicatie. Omdat er
een nieuwe datastructuur moet komen en de logica van het oude systeem nauw verbonden is met de
datastructuur is het niet mogelijk om de oude code opnieuw te kunnen gebruiken. Dit platform moet een
grote hoeveelheid potentiële kleine klanten kunnen ondersteunen naast de grotere klanten van Snakeware.
\todo[inline]{misschien het project een code naam geven zodat er makkelijk naar gewezen kan worden}
