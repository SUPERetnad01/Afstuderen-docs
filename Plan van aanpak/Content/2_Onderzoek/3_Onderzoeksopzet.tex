\section{Onderzoeksopzet}
Het onderzoek wordt opgezet door gebruik te maken van de  methode van Nel Verhoeven \Parencite{Verhoeven}.
De methode van Verhoeven bestaat uit de volgende 4 fases:
\whitespace
1. Ontwerpen \\
2. Gegevens verzamelen \\
3. Analyseren \\
4. Evalueren en adviseren \\
\whitespace
Door middel van de fases wordt het onderzoek opgesteld, dit wordt gerepresenteerd door de verschillende hoofdstukken in het onderzoek: \\
- De inleiding en Onderzoeksopzet zijn onderdeel van de eerste fase van de methode van Verhoeven (ontwerpen).
Hier wordt de context van het onderzoek in beeld gebracht en hoe het onderzoek uitgevoerd gaat worden. \\ 
- De gegevens worden verzameld en geanalyseerd (fase 2 en 3 van Verhoven \Parencite{Verhoeven}) in het hoofdstuk resultaten.
Dit wordt gedaan door middel van deelvragen en onderzoeksmethodes die op zijn gezet in de eerste fase.\\ 
- In de laatste fase (Evalueren en adviseren) wordt gedaan in het hoofdstuk conclusie, hier wordt er een conclusie getrokken uit de resultaten die in de vorige hoofdstukken gekomen zijn.
% door middel van deze fases worden de volgende hoofdstukken opgesteld: \\
% - Hoofdstuken Inleiding en Onderzoeksmethoden en -opzet zijn onderdeel van de eerste phase van de methode van Verhoeven (ontwerpen).\\
% - De gegevens worden verzameld in het hoofdstuk resultaten, hier wordt getoont hoe er is gekomen tot het resultaat vanuit de eerdere hoofdstukken. \\
% - In het laatste hoofdstuk wordt er een conclusie getrokken met daar bij een reflectie hoe het onderzoek volgende keer beter had kunnen gaan. \\
