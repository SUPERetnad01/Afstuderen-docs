
\section{Onderzoeksmethoden}
Om de betrouwbaarheid van het onderzoek te valideren wordt er gebruik gemaakt van verschillende onderzoeksmethoden.
Deze onderzoeksmethoden worden beschreven door HBO-I (https://ictresearchmethods.nl/Methods). \todo[inline]{fix apa}
\whitespace
% Deze deelvraag wordt onderzocht door een \textbf{stakeholders analyse} uit te voeren.
% De stakeholders worden geindentificeerd door middel van \textbf{interviews} met de betrokkende partijen van het project.
% Daarna worden de stakeholders geprioriteerd op basis van belang en invloed.
\textit{\textbf{Deelvraag 1:} \SubquestionOne} \\
Bij deze deelvraag wordt er onderzocht wat het nu de huidige klachten zijn van het Snakeware Cloud platform te werken.
Doormiddel van \textbf{geconstructureerde expertinterviews} wordt er onderzocht waar de huidige klachten zitten, en wat het veroorzaakt.
Na het expertinterview wordt er een \textbf{task analyse} gedaan om de knelpunten in beeld te krijgen.
De interviews worden gedaan met medewerkers van Snakeware en de stakeholders, hierdoor kunnen alle huidige klachten in beeld worden gebracht.
\todo[inline]{validiteit: Er zijn expertinterviews gedaan een task analyse.Dit is valide omdat de medewerkers van Snakeware veel kennis hebben van het systeem}
\todo[inline]{betrouwbaarheid: de interview vragen zijn van te voren opgesteld en de mensen waar mee het interview mee gedaan wordt zijn benoemd}
\whitespace
\textit{\textbf{Deelvraag 2:} \SubquestionTwo} \\
Omdat Snakeware Cloud niet het enige CMS platform is wordt er onderzocht doormiddel van \textbf{avaiable product analysis} wat de concurenten gebruiken kwa techniek.
Daarnaast wordt er door middel van \textbf{literatuur studies} onderzocht hoe deze technieken werken en mogelijk zouden kunnen passen in het systeem.
Voor de avaiable product analysis wordt er gebruikt gemaakt van een voor afgesproken lijst van concurenten/opensource projecten.
Hier uit komt een lijst van technieken / architecturen deze informatie wordt gerangschikt opbasis van relevantie.
Deze infromatie wordt gebruikt om de stakeholders te informeren over de non-functional requirements van het systeem.
\todo[inline]{validiteit: Er wordt gebruik gemaakt van een Avalilible product analyse en literatuur studies.}
\todo[inline]{betrouwbaarheid: De wordt verteld dat er gebruik wordt gemaakt van een lijst van concurenten die zou in een volgedn onderzoek herhaald kunnen worden.}
\whitespace
\textit{\textbf{Deelvraag 3:} \SubquestionThree} \\
Deze deelvraag wordt onderzocht door de stakeholders te interviewen door middel van een \textbf{geconstructureerd interview}, de vragen voor dit interview worden van te voren opgesteld en gevalideerd met de product owner/bedrijfsbegeider.
Vervolgens worden de wensen en eisen in beeld gebracht door middel van \textbf{Explore user requirements}.
\todo[inline]{validiteit: Er wordt gebruik gemaakt van geconstructureerd inteview Explore user requirements, verder worden de vragen gevalideerd door mensen. }
\todo[inline]{betrouwbaarheid: omdat er een lijst van vragen is gemaakt kan dit makkelijk opnieuw gevoerd worden.}
\whitespace
\textit{\textbf{Deelvraag 4:} \SubquestionFour} \\
Deze deelvraag beantwoord de prioritatie van de requirements,
er wordt samen met de bedrijfsbegeider een priortering op gezet (\textbf{requirements prioritisation}) om de duur en scope in gedachte te houden met het maken van de afstudeeropdracht.
De requirement prioritisation wordt gedaan doormiddel van de \textbf{MoSCoW} methode \todo[inline]{bron naar apa}.
\todo[inline]{validiteit: er wordt gebruikt van een veelgebruikte methode, prioritatie wordt gevalideerd.}
\todo[inline]{betrouwbaarheid: bekende methodes de requirements worden gelevered als gedeelte van deze deelvraag}
