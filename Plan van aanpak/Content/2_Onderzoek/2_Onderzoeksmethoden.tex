\section{Onderzoeksmethoden}
In dit hoofdstuk worden de onderzoeksmethoden beschreven die worden gebruikt tijdens het onderzoek.

\whitespace[2]
\textit{\textbf{Deelvraag 1:} \SubquestionOne} \\
Om te weten wat de eisen en wensen zijn voor het systeem is het belangrijk dat je weet voor wie je het maakt.
Daarom worden de stakeholders van het project worden in kaart gebracht door middel van een \textbf{stakeholdersanalyse}.
Als de stakeholders zijn geïdentificeerd worden ze daar na geprioriteerd op basis van invloed en belang.

\whitespace[2]
\textit{\textbf{Deelvraag 2:} \SubquestionTwo} \\
Om de huidige problemen van het Snakeware cloud platform in beeld te brengen is het belangrijk dat er gekeken naar de huidige architectuur van snakeware cloud.
Daarom wordt er door middel van \textbf{IT architecture sketching} met samenwerking van het R\&D team de huidige architectuur in beeld gebracht.
Hier uit wordt een lijst met problemen verzameld die de huidige architectuur nu heeft, en wordt ter ondersteuning gebruikt van deelvraag 3 (\textit{\SubquestionThree}).

\whitespace[2]
\textit{\textbf{Deelvraag 3:} \SubquestionThree} \\
Bij deze deelvraag wordt er onderzocht wat nu de huidige knelpunten zijn van het Snakeware Cloud platform.
Door middel van gestructureerde \textbf{expertinterviews} wordt er onderzocht waar de huidige klachten zitten, en wat het veroorzaakt.
Na het expertinterview wordt er een \textbf{task analyse} gedaan om de knelpunten in beeld te krijgen.
De interviews worden gedaan met de stakeholders, hierdoor kunnen alle huidige klachten in beeld worden gebracht.

\whitespace[2]
\textit{\textbf{Deelvraag 4:} \SubquestionFour} \\
Deze deelvraag wordt onderzocht door de stakeholders te interviewen door middel van gestructureerde \textbf{interviews}.
De vragen voor dit interview worden van tevoren opgesteld en gevalideerd met de product owner en de bedrijfsbegeleider.
Vervolgens worden de wensen en eisen in beeld gebracht door middel van \textbf{explore user requirements}.

\whitespace[2]
\textit{\textbf{Deelvraag 5:} \SubquestionFive} \\
Deze deelvraag beantwoord de prioriteit van de verschillende requirements.
Na dat alle requirements zijn verzameld worden er prioriteiten aan toegekend door middel van \textbf{requirements prioritization}.
Vervolgens worden ze  gecategoriseerd door middel van de MoSCoW methode \Parencite{MoSCoW}.
Als de requirements zijn geprioriteerd zijn dan wordt er samen met de bedrijfsbegeleider bepaald welke requirements gerealiseerd worden, dit wordt gedaan om de afstudeeropdracht scope haalbaar te maken.
