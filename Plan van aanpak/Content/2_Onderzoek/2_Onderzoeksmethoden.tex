\section{Onderzoeksmethoden}
In dit hoofdstuk worden de onderzoeksmethoden beschreven en de gebruikte onderzoeksmethoden worden bij elke deelvraag benoemd.
Dit wordt gedaan om de betrouwbaarheid van het onderzoek te valideren.
\whitespace
\textit{\textbf{Deelvraag 1:} \SubquestionOne} \\
De stakeholders worden eerst geïdentificeerd door middel van een \textbf{stakeholdersanalyse} en vervolgens worden ze geprioriteerd op basis van belang en invloed.
\whitespace
\textit{\textbf{Deelvraag 2:} \SubquestionTwo} \\
Wat is de huidige softwarearchitectuur van Snakeware Cloud?
Omdat Snakeware Cloud zijn huidige architectuur problemen oplevert, is het belangrijk dat deze fouten niet op nieuw worden gemaakt.
Daarom door middel van \textbf{IT architecture sketching} met samenwerking van het R\&D team wordt het systeem in beeld gebracht.
Hier uit komt  een lijst van requirements, die gebruikt worden tijdens het infomeren van de stakeholders.
% Omdat Snakeware Cloud zijn huidige archituur problemen opleverd is het belanrijk dat deze fouten niet op nieuw worden gemaakt.
% Daarom doormiddel van \textbf{IT architecture sketching} met samenwerking van het R\&D team wordt het systeem in beeld gebracht.
% Hier uit komt een lijst van requirements, die gebruikt worden tijdens het infomeren van de stakeholders.
\whitespace
\textit{\textbf{Deelvraag 3:} \SubquestionThree} \\
Bij deze deelvraag wordt er onderzocht wat nu de huidige knelpunten zijn van het Snakeware Cloud platform.
Doormiddel van gestructureerde \textbf{expertinterviews} wordt er onderzocht waar de huidige klachten zitten, en wat het veroorzaakt.
Na het expertinterview wordt er een \textbf{task analyse} gedaan om de knelpunten in beeld te krijgen.
De interviews worden gedaan met de stakeholders, hierdoor kunnen alle huidige klachten in beeld worden gebracht.
\whitespace
% \todo[inline]{validiteit: Er zijn expertinterviews gedaan een task analyse.Dit is valide omdat de medewerkers van Snakeware veel kennis hebben van het systeem}
% \todo[inline]{betrouwbaarheid: de interview vragen zijn van te voren opgesteld en de mensen waar mee het interview mee gedaan wordt zijn benoemd}
% \todo[inline]{dit werdt aangeraden door stefan hier uit zou je stuk concretere vraag krijgen en resultaat.}
% Omdat Snakeware Cloud niet het enige \gls{CMS} platform is wordt er onderzocht doormiddel van \textbf{available product analysis} wat de concurenten gebruiken kwa techniek.
% Daarnaast wordt er door middel van \textbf{literatuur studies} onderzocht hoe deze technieken werken en mogelijk zouden kunnen passen in het systeem.
% \todo[inline]{hier nog een keertje naar kijken}
% Voor de available product analysis wordt er gebruikt gemaakt van een voor afgesproken lijst van concurenten/opensource projecten.
% Hier uit komt een lijst van technieken / architecturen deze informatie wordt gerangschikt opbasis van relevantie.
% Deze infromatie wordt gebruikt om de stakeholders te informeren over de non-functional requirements van het systeem.
% \todo[inline]{validiteit: Er wordt gebruk gemaakt van IT architure.}
% \todo[inline]{betrouwbaarheid: Er wordt verteld dat het gedaan wordt met het R&D team}
\textit{\textbf{Deelvraag 4:} \SubquestionFour} \\
Deze deelvraag wordt onderzocht door de stakeholders te interviewen door middel van gestructureerde \textbf{interviews}. 
De vragen voor dit interview worden van tevoren opgesteld en gevalideerd met de product owner/bedrijfsbegeleider.
Vervolgens worden de wensen en eisen in beeld gebracht door middel van \textbf{explore user requirements}.
% \todo[inline]{validiteit: Er wordt gebruik gemaakt van geconstructureerd inteview explore user requirements, verder worden de vragen gevalideerd door mensen. }
% \todo[inline]{betrouwbaarheid: omdat er een lijst van vragen is gemaakt kan dit makkelijk opnieuw gevoerd worden.}
\whitespace
\textit{\textbf{Deelvraag 5:} \SubquetionFive} \\
Deze deelvraag beantwoord de prioritatiet van de verschillende requirements.
Na dat alle requirements zijn verzameld worden er prioriteiten aan toegekend door middel van \textbf{requirements prioritization}.
Vervolgens worden ze  gecategoriseerd van doormiddel van de \textbf{MoSCoW} methode \Parencite{MoSCoW}. 
Als de requirements zijn geprioriteerd zijn dan wordt er samen met de bedrijfsbegeleider gekeken naar wat voor requirements in de scope afstudeeropdracht past.
% De requirements die niet mee genomen kunnen worden kunnen mogelijk na de afstudeeropdracht toegevoegt worden.
% \todo[inline]{validiteit: er wordt gebruikt van een veelgebruikte methode, prioritatie wordt gevalideerd.}
% \todo[inline]{betrouwbaarheid: bekende methodes de requirements worden gelevered als gedeelte van deze deelvraag}
