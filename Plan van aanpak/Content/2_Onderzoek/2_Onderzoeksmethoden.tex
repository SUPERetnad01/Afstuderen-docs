\section{Onderzoeksmethoden}
In dit hoofdstuk worden de onderzoeksmethoden beschreven die worden gebruikt tijdens het onderzoek.

\whitespace[2]
\textit{\textbf{Deelvraag 1:} \SubquestionOne} \\
De stakeholders van het project worden in gebracht door middel van een \textbf{stakeholdersanalyse}.
Als de stakeholders zijn geïdentificeerd worden ze daar na geprioriteerd op basis van invloed en belang.

\whitespace[2]
\textit{\textbf{Deelvraag 2:} \SubquestionTwo} \\
\todo[inline]{Wachten op feedback van stefan}
% - hoet ziet het huidige structuur er uit?
% - zodat we met informatie van de huidige structuur bezig kunnen gan naar de task analyse om te bekijken waar in het systeem nu de huidige problemen naar toe te komen
%
% Omdat Snakeware Cloud zijn huidige architectuur problemen oplevert, is het belangrijk dat deze fouten niet op nieuw worden gemaakt.
% Daarom door middel van \textbf{IT architecture sketching} met samenwerking van het R\&D team wordt het systeem in beeld gebracht.
% Hier uit komt  een lijst van requirements, die gebruikt worden tijdens het informeren van de stakeholders.
%
\whitespace[2]
\textit{\textbf{Deelvraag 3:} \SubquestionThree} \\
Bij deze deelvraag wordt er onderzocht wat nu de huidige knelpunten zijn van het Snakeware Cloud platform.
Door middel van gestructureerde \textbf{expertinterviews} wordt er onderzocht waar de huidige klachten zitten, en wat het veroorzaakt.
Na het expertinterview wordt er een \textbf{task analyse} gedaan om de knelpunten in beeld te krijgen.
De interviews worden gedaan met de stakeholders, hierdoor kunnen alle huidige klachten in beeld worden gebracht.

\whitespace[2]
\textit{\textbf{Deelvraag 4:} \SubquestionFour} \\
Deze deelvraag wordt onderzocht door de stakeholders te interviewen door middel van gestructureerde \textbf{interviews}. 
De vragen voor dit interview worden van tevoren opgesteld en gevalideerd met de product owner en de bedrijfsbegeleider.
Vervolgens worden de wensen en eisen in beeld gebracht door middel van \textbf{explore user requirements}.

\whitespace[2]
\textit{\textbf{Deelvraag 5:} \SubquestionFive} \\
Deze deelvraag beantwoord de prioriteit van de verschillende requirements.
Na dat alle requirements zijn verzameld worden er prioriteiten aan toegekend door middel van \textbf{requirements prioritization}.
Vervolgens worden ze  gecategoriseerd van door middel van de MoSCoW methode \Parencite{MoSCoW}. 
Als de requirements zijn geprioriteerd zijn dan wordt er samen met de bedrijfsbegeleider gekeken naar wat voor requirements in de scope afstudeeropdracht past.
