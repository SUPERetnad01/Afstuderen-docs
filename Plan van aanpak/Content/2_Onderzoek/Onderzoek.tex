\chapter{Onderzoek}
In de onderzoeksfase wordt er onderzocht wat de functionele,niet functionele requirements en randvoorwaarden zijn die
worden gesteld aan het softwaresysteem. Het onderzoek wordt door middel van de methode van
Verhoeven uitgevoerd. Er wordt daarnaast gebruikgemaakt van de onderzoekmethoden beschreven door
HBO-I (https://ictresearchmethods.nl/Methods) \todo[inline]{fix apa}.
Om het onderzoek te ondersteuen is er een \textbf{stakeholder analyse} gedaan.
De stakeholders zijn eerst geindentificeerd door middel van \textbf{onderzoeks methde} en vervolgens zijn geprioriteerd op basis van belang en invloed.
de stakeholders analyse is te vinden vinden in hoofdstuk 1.5. \todo[inline]{maak reference ect}
% \todo[inline]{examen commisie \emph{Tip: voor je onderzoek kan het misschien ook niet zijn om te kijken naar systeem van de concurenten of open-source oplossingen,
% 		daarin zitten vast ook bruikbare requirements. Zo'n avaiable product analysis past mooi in je technisch ontwerp!}}

% \\textbf{Deelvraag 5:} \SubquestionFive  \\
\section{Hoofd- en deelvragen}
In dit hoofdstuk worden de hoofd- en deelvragen behandeld.
Eerst wordt de hoofdvraag behandeld en daarna de deelvragen.
\whitespace
\textbf{Hoofdvraag:} Wat zijn de requirements die worden gesteld aan een contentmanagementsysteem waarmee
Snakeware webapplicaties ook aan kleinere klanten kan aanbieden?
\whitespace
% Voor deze hoofdvraag zijn de volgende deelvragen opgesteld:
\textbf{Deelvraag 1:} \SubquestionOne   \\
\textbf{Deelvraag 2:} \SubquestionTwo   \\
\textbf{Deelvraag 3:} \SubquestionThree \\
\textbf{Deelvraag 4:} \SubquestionFour  \\


\section{Onderzoeksmethoden}
Om de betrouwbaarheid van het onderzoek te valideren wordt er gebruik gemaakt van verschillende onderzoeksmethoden.
Deze onderzoeksmethoden worden beschreven door HBO-I (https://ictresearchmethods.nl/Methods). \todo[inline]{fix apa}
\whitespace
% Deze deelvraag wordt onderzocht door een \textbf{stakeholders analyse} uit te voeren.
% De stakeholders worden geindentificeerd door middel van \textbf{interviews} met de betrokkende partijen van het project.
% Daarna worden de stakeholders geprioriteerd op basis van belang en invloed.
\textit{\textbf{Deelvraag 1:} \SubquestionOne} \\
Bij deze deelvraag wordt er onderzocht wat het nu de huidige klachten zijn van het Snakeware Cloud platform te werken.
Doormiddel van \textbf{geconstructureerde expertinterviews} wordt er onderzocht waar de huidige klachten zitten, en wat het veroorzaakt.
Na het expertinterview wordt er een \textbf{task analyse} gedaan om de knelpunten in beeld te krijgen.
De interviews worden gedaan met medewerkers van Snakeware en de stakeholders, hierdoor kunnen alle huidige klachten in beeld worden gebracht.
\todo[inline]{validiteit: Er zijn expertinterviews gedaan een task analyse.Dit is valide omdat de medewerkers van Snakeware veel kennis hebben van het systeem}
\todo[inline]{betrouwbaarheid: de interview vragen zijn van te voren opgesteld en de mensen waar mee het interview mee gedaan wordt zijn benoemd}
\whitespace
\textit{\textbf{Deelvraag 2:} \SubquestionTwo} \\
Omdat Snakeware Cloud niet het enige CMS platform is wordt er onderzocht doormiddel van \textbf{avaiable product analysis} wat de concurenten gebruiken kwa techniek.
Daarnaast wordt er door middel van \textbf{literatuur studies} onderzocht hoe deze technieken werken en mogelijk zouden kunnen passen in het systeem.
Voor de avaiable product analysis wordt er gebruikt gemaakt van een voor afgesproken lijst van concurenten/opensource projecten.
Hier uit komt een lijst van technieken / architecturen deze informatie wordt gerangschikt opbasis van relevantie.
Deze infromatie wordt gebruikt om de stakeholders te informeren over de non-functional requirements van het systeem.
\todo[inline]{validiteit: Er wordt gebruik gemaakt van een Avalilible product analyse en literatuur studies.}
\todo[inline]{betrouwbaarheid: De wordt verteld dat er gebruik wordt gemaakt van een lijst van concurenten die zou in een volgedn onderzoek herhaald kunnen worden.}
\whitespace
\textit{\textbf{Deelvraag 3:} \SubquestionThree} \\
Deze deelvraag wordt onderzocht door de stakeholders te interviewen door middel van een \textbf{geconstructureerd interview}, de vragen voor dit interview worden van te voren opgesteld en gevalideerd met de product owner/bedrijfsbegeider.
Vervolgens worden de wensen en eisen in beeld gebracht door middel van \textbf{Explore user requirements}.
\todo[inline]{validiteit: Er wordt gebruik gemaakt van geconstructureerd inteview Explore user requirements, verder worden de vragen gevalideerd door mensen. }
\todo[inline]{betrouwbaarheid: omdat er een lijst van vragen is gemaakt kan dit makkelijk opnieuw gevoerd worden.}
\whitespace
\textit{\textbf{Deelvraag 4:} \SubquestionFour} \\
Deze deelvraag beantwoord de prioritatie van de requirements,
er wordt samen met de bedrijfsbegeider een priortering op gezet (\textbf{requirements prioritisation}) om de duur en scope in gedachte te houden met het maken van de afstudeeropdracht.
De requirement prioritisation wordt gedaan doormiddel van de \textbf{MoSCoW} methode \todo[inline]{bron naar apa}.
\todo[inline]{validiteit: er wordt gebruikt van een veelgebruikte methode, prioritatie wordt gevalideerd.}
\todo[inline]{betrouwbaarheid: bekende methodes de requirements worden gelevered als gedeelte van deze deelvraag}

\section{Onderzoeksopzet}
Het onderzoek wordt opgezet door gebruik te maken van de  methode van Nel Verhoeven \Parencite{Verhoeven}.
De methode van Verhoeven bestaat uit de volgende 4 fases:
\whitespace
1. Ontwerpen \\
2. Gegevens verzamelen \\
3. Analyseren \\
4. Evalueren en adviseren \\
\whitespace
Door middel van de fases wordt het onderzoek opgesteld, dit wordt gerepresenteerd door de verschillende hoofdstukken in het onderzoek: \\
- De inleiding en Onderzoeksopzet zijn onderdeel van de eerste fase van de methode van Verhoeven (ontwerpen).
Hier wordt de context van het onderzoek in beeld gebracht en hoe het onderzoek uitgevoerd gaat worden. \\ 
- De gegevens worden verzameld en geanalyseerd (fase 2 en 3 van Verhoven \Parencite{Verhoeven}) in het hoofdstuk resultaten.
Dit wordt gedaan door middel van deelvragen en onderzoeksmethodes die op zijn gezet in de eerste fase.\\ 
- In de laaste fase (Evalueren en adviseren) wordt gedaan in het hoofdstuk conclusie, hier wordt er een conclusie getrokken uit de resultaten die in de vorige hoofdstukken gekomen zijn.
% door middel van deze fases worden de volgende hoofdstukken opgesteld: \\
% - Hoofdstuken Inleiding en Onderzoeksmethoden en -opzet zijn onderdeel van de eerste phase van de methode van Verhoeven (ontwerpen).\\
% - De gegevens worden verzameld in het hoofdstuk resultaten, hier wordt getoont hoe er is gekomen tot het resultaat vanuit de eerdere hoofdstukken. \\
% - In het laatste hoofdstuk wordt er een conclusie getrokken met daar bij een reflectie hoe het onderzoek volgende keer beter had kunnen gaan. \\

\section{Tijdslijn}
De onderzoeks fase loopt van het being van week 41 (9 oktober 2023) tot in het midde van de werk week 46 (15 november 2023).
Het onderzoekverslag wordt opgezet in het begin van de week (41). In de rest van de week worden de onderzoeksmethoden opgezet voor uitvoering.
In week 42 tm/ week 44 worden de onderzoeksmethoden uitgevoerd en resultaten verzameld. In week 45 wordt er geconcludeerd en gereflecteerd.
Tijdens de verslag legging van het onderzoek zal er met regelmaat contact gemaakt worden met de docent begeleider om het proces naar een goed einde te leiden.
Als het onderzoek is afgerond kan er  op basis van de requirements een ontwerp gemaakt worden van het systeem.
\todo[inline]{wachten voor een bericht van school aangezien het een hele periode is, want de data lijk veel te kort. }


\section{Bronnen bij onderzoeksonderdelen}
\todo[inline]{is dit hooftstuk nog nodig?}
\begin{tabular}{ | p{5cm} | p{9cm} | }
	\hline
	\textbf{Deelvraag} & \textbf{Bronnen}                                                                                                                               \\
	\hline
	\SubquestionOne    & Door middel van \textbf{interviews} van de collega's van Snakeware en de stakeholders worden de pijn punten in beeld gebracht.
	Dit wordt vervolgens gemoduleerd doormiddel van \textbf{task analysis}. \\
	\hline
	\SubquestionTwo    & De bronnen bij deze deelvraag zijn de open-source oplossigen van andere concurenten, en informatie op het internet.                              \\
	\hline
	\SubquestionThree  & Door \textbf{interviews} te doen met de stakeholders te doen kunnend de eisen en wensen in kaart gebracht worden.                                                                                             \\
	\hline
	\SubquestionFour   & Als de requirements geprioriteerd zijn worden ze geferifeerd met de bedrijfsbegeider en de stakeholders.                            \\
	\hline
\end{tabular}

