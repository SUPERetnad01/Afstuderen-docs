\chapter{Projectactiviteiten}
In dit hoofdstuk wordt behandeld de projectactiviteiten in de afstudeerfase.
Deze activiteiten omvatten het opstellen van het plan van aanpak, het uitvoeren van onderzoek, de productontwikkeling (met inclusie van het technisch verslag), de eindpresentatie en de demonstratie.
Daarnaast zullen ook de doorlopende activiteiten worden besproken.
\section{Plan van Aanpak}
% \todo[inline]{wat to do}
Voor dit project wordt een plan van aanpak opgesteld om het project helder en concreet te definiëren.
Dit plan van aanpak dient als een leidraad voor zowel de docentbegeleider als Snakeware, waarin wordt beschreven hoe het project wordt aangepakt.
Tevens wordt de aanpak van het onderzoek in detail uiteengezet binnen dit plan van aanpak, waardoor er direct aan het onderzoek kan worden gewerkt.
\section{Onderzoek}
Het onderzoek wordt uitgevoerd over 5 weken, hierbij worden de verschillende deelvragen toegewezen aan hun eigen activiteit.
Deze activiteiten tot betrekking van de deelvragen zijn: \\
- Stakeholders analyse doen binnen Snakeware. \\ 
- Onderzoeken huidige architectuur van Snakeware Cloud. \\
- Knelpunten onderzoeken van Snakeware Cloud. \\ 
- Explore user requirements uitvoeren bij de stakeholders. \\
- Requirement prioratization uitvoeren. \\
Daarnaast wordt er ook een conclusie en discussie geschreven.

% \begin{graphic}
% 	\captionsetup{type=table}
% 	\begin{tabular}{|5cm|c|}
% 		\hline
% 		\textbf{Project activiteit}                                            & \textbf{Bescrijving} \\
% 		\hline
% 		Stakeholders analyse uitvoeren (deelvr. 1)                             & De stakeholders analyse word binnen Snakeware uitgevoerd.                 \\
% 		\hline
% 		IT architectuur uitvoeren (deelvr. 2)                                  & test                 \\
% 		\hline
% 		Task analyse en gestructrueerde expertinterviews uitvoeren (deelvr. 3) & test                 \\
% 		\hline
% 		Explore user requirements bij de stakeholders (deelvr. 4)              & test                 \\
% 		\hline
% 		Requirements prioratizeren (deelvraag. 5)                              & test                 \\
% 		\hline
% 		conclusie schrijven                                                    & test                 \\
% 		\hline
% 		Discussie schrijven                                                    & test                 \\
% 	\end{tabular}
% 	\caption{Deadlines en conceptversie inlever momenten}
% 	\label{tab:OnderzoekProjectActiviteiten}
% \end{graphic}
% Het onderzoek wordt uitgevoerd volgens de methode van Verhoeven, zoals beschreven in "Wat is onderzoek" \Parencite{Verhoeven}.
% Voor uitgebreidere informatie met betrekking tot het onderzoek, wordt er verwezen naar Hoofdstuk \ref{chap:Onderzoek}.
\section{Product}
In de realisatieperiode van het product zijn er 2 weken gepland om de basis van systeem te ontwerpen.
Na de basis van het ontwerp opgezet is worden er 9 sprints geplant, waarbij elke sprint 2 weken lang duurt.
Tijdens deze sprints wordt voornamelijk de code geschreven en getest.
Door de sprints heen wordt het technisch verslag opgezet en het daarbij behorende 4 + 1 architectuurmodel.
Hierdoor bestaat het realiseren van het product uit de volgende activiteiten: \\
- Opstellen van technische documentatie\\
- Ontwerp opzetten \\
- Testen \\
- 4 + 1 architectuurmodel opzetten\\
- Sprints (coderen)
\section{Eindpresentatie}
Aan het einde van de afstudeerperiode wordt een afsluitende presentatie gehouden voor de Bedrijfsbegeleider en andere collega's bij Snakeware.
Daarnaast wordt eveneens een eindpresentatie verzorgd voor de docentbegeleider en de examinatoren.
Dit wordt vertaald naar de volgende activiteiten: \\
- Voorbereiding presentatie \\
- Presentaties \\
- Demonstratie product
\section{Ondersteuning}
Tijdens de afstudeerperiode zal er ondersteuning gevraagd worden om de afstudeeropdracht in een goede baan te leiden. 
Er zal voortdurend contact onderhouden met de Bedrijfsbegeleider, waarbij er een wekelijkse feedbackmomenten plaatsvindt om de voortgang van het project te bespreken en toekomstige stappen te bepalen.
Ook zal er per week een keer afgesproken worden met de docentbegeleider om de stand van zaken te bespreken.
Daarnaast wordt er een keer per maand een technisch overleg georganiseerd waarbij kennis wordt gedeeld met betrekking tot technische uitdagingen binnen Snakeware.
Dit wordt vertaald naar de volgende activiteiten: \\
- Contact met de Bedrijfsbegeleider. \\
- Contact met de docentbegeleider. \\
- Technisch overleg
