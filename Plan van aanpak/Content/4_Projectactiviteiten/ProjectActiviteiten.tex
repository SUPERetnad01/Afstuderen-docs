\chapter{Projectactiviteiten}
Dit hoofdstuk behandelt de projectactiviteiten in de afstudeerfase.
Deze activiteiten omvatten het opstellen van het plan van aanpak, het uitvoeren van onderzoek, de productontwikkeling (met inclusie van het technisch verslag), de eindpresentatie en de demonstratie.
Daarnaast zullen ook de doorlopende activiteiten worden besproken.
\section{Plan van Aanpak}
Voor dit project wordt een plan van aanpak opgesteld om het project helder en concreet te definiëren.
Dit plan van aanpak dient als een leidraad voor zowel de docentbegeleider als Snakeware, waarin wordt beschreven hoe het project wordt aangepakt.
Tevens wordt de aanpak van het onderzoek in detail uiteengezet binnen dit plan van aanpak, waardoor er direct aan het onderzoek kan worden gewerkt.
\section{Onderzoek}
Het onderzoek wordt uitgevoerd volgens de methode van Verhoeven, zoals beschreven in "Wat is onderzoek" \Parencite{Verhoeven}.
Voor uitgebreidere informatie met betrekking tot het onderzoek, wordt er verwezen naar Hoofdstuk \ref{chap:Onderzoek}.
\section{Product}
\todo[inline]{aantal sprints wordt tijdens het planning hoofdstuk ingevuld}
In de realisatieperiode van het product zijn X sprints gepland, waarbij elke sprint een tijdsbestek van 2 weken bestaat.
Deze sprints worden ingezet om de diverse aspecten van de realisatiefase aan te pakken.
De realisatiefase omvat de volgende onderdelen: ontwerp, ontwikkeling, testen, en terugkoppeling, waarbij terugkoppeling wordt verkregen door middel van code reviews en reflecties.
\section{Eindpresentatie}
Aan het einde van de afstudeerperiode wordt een afsluitende presentatie gehouden voor de Bedrijfsbegeleider en andere collega's bij Snakeware.
Daarnaast wordt eveneens een eindpresentatie verzorgd voor de docentbegeleider en de examinatoren.
\section{Doorlopende activiteiten}
Tijdens de afstudeerperiode zijn er verschillende activiteiten die lopen van begin tot eind van het project.
Er zal voortdurend contact onderhouden worden met de Bedrijfsbegeleider, waarbij wekelijks feedbackmomenten plaatsvinden om de voortgang van het project te bespreken en toekomstige stappen te bepalen.
Eveneens zal er regelmatig contact zijn met de docentbegeleider.
Daarnaast wordt er een keer per maand een technisch overleg georganiseerd waarbij kennis wordt gedeeld met betrekking tot technische uitdagingen binnen Snakeware.
