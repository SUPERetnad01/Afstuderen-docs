\subsection{Deelvraag 3: Knelpunten}
Om het systeem te kunnen ontwikkelen moeten er requirements aan het systeem gesteld worden.
Deze requirements moeten op basis van de eisen en wensen van de stakeholders gemaakt worden.
Hieruit zal een lijst requirements komen waar mee het systeem gerealiseerd wordt.
Daarom is de volgende deelvraag gemaakt:

\begin{center}
    \textit{\SubquestionThree}
\end{center}

\whitespace[0.2]
Om deze deelvraag te beantwoorden wordt er een semi-gestructureerd \textbf{expertinterview} gehouden.
Binnen Snakeware zijn er meerdere mensen die geschikt zijn om de knelpunten van het Snakeware cloud platform te kunnen aankaarten.
De volgende mensen worden de interviews afgenomen:
\begin{itemize}
    \item[-] Janny Reitsma (Service desk lead?): Reitsma heeft veel inzicht in waar de huidige klanten van snakeware tegen aanlopen.
        Verder krijgt ze alle klachten van de klanten van Snakeware mee en weet ze waar de huidige klanten van Snakeware behoefte aan hebben.
    \item[-] Rob Douma (Product owner van meerdere projecten): Douma werkt aan meerdere projecten als product owner en weet veel van Snakeware cloud.
        Hij heeft veel technische kennis over het platform en kan goed in beeld brengen wat de huidige technische imitaties zijn van het platform.
\end{itemize}

\whitespace
Er is overwogen om Hans Hoomans (CEO) en Johan Nieuwehuis (CTO) te interviewen om de huidige knelpunten in beeld te brengen.
Dit is uiteindelijk niet gedaan vanwege de tijd die beschikbaar is voor het onderzoek.
% Na de expert interviews zal er een taak analyse uitgevoerd worden om de werkwijze van het CMS in beeld te krijgen.
\todo[inline]{Ik weet niet of de titles kloppen van Rob en Janny}
