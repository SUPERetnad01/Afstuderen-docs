\subsection{Deelvraag 4: Requirements}
Om het systeem te kunnen ontwikkelen moeten er requirements aan het systeem gesteld worden.
Deze requirements moeten op basis van de eisen en wensen van de stakeholders gemaakt worden.
Hieruit zal een lijst requirements komen waar mee het systeem gerealiseerd wordt.
Daarom is de volgende deelvraag gemaakt:

\begin{center}
 \textit{\SubquestionFour}
\end{center}

\whitespace[0.2]
Om deze deelvraag te beantwoorden wordt er gebruik gemaakt van \textbf{explore user requirements}.
De communicatiemethode met de stakeholders wordt bepaald op basis van hun positie binnen het project door middel van de stakeholder matrix.
Voor de sleutelfiguren worden er semigestructureerde \textbf{interviews} gehouden om genoeg vrijheid te geven tijdens de gesprekken om dieper op vragen in te gaan.
Daarnaast worden er met de geïnteresseerde een \textbf{focus group} gepland om hier met de betrekkende stakeholders meerdere onderwerpen te bespreken die belangrijk zijn voor het project.
Voor de focus groep zal er gebruik gemaakt worden van een aantal voorbereide vragen om de focus groep in een goede richting te sturen.
Als de eisen en wensen zijn bepaald door middel van de focus group en interviews worden ze genoteerd zodat ze in de volgende deelvraag geprioriteerd kunnen worden.
\todo[inline]{Verwachte resultaat?}
