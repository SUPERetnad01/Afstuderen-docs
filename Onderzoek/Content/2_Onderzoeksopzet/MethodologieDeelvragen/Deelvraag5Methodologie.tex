\subsection{Deelvraag 5: Prioritering}
Na het opstellen van een lijst met requirements als resultaat van deelvraag 4.
Deze lijst is echter nog niet bruikbaar, aangezien deze niet is geprioriteerd.
Om de prioriteiten van de requirements vast te stellen, wordt de volgende deelvraag geintroduceerd.

\begin{center}
	\textit{\SubquestionFive}
\end{center}

\whitespace[0.2]
Dit wordt gedaan door middel van \textbf{requirements prioritization} zal er verschillende prioriteit niveaus toegekend worden aan de requirements.
Deze niveaus worden in beeld gebracht door middel van MoSCoW-methode \Parencite{MoSCoW}.
%
% tevredenheidsscore [1,2, \dots, 5]
% ontevredenheidscore [1,2 \dots, 5]
% duur (1,2,3,5,8)
% Stakeholder positie [1,2,3,4]

\whitespace
Om de prioritering te bepalen wordt er gebruik gemaakt van een formule (zie formule \ref{eq:prioritization}), in de ze formule wordt de volgnede aspecten in mee genomen.
\begin{enumerate}
	\item[-] tevredenheidsscore (TS) [1,2,\ldots,5]: dit is de waarde die door de stakeholder gegeven wordt als de requirement geimplementeerd wordt
	\item[-] ontevredenheidscore (OS) [1,2,\dots,5]: Dit is de waarde die door de stakeholder gegeven wordt wanneer het niet geimplementeerd wordt.
	\item[-] stakeholder invloed positie (SIP) [1,2,3,4]: Op basis van de stakeholders matrix wordt er een waarde aan een stakeholder groep toegekend 1 voor sleutelfiguren, 2 voor beinvloeder, 3 voor geintreceerde en 4 voor toeschouwer.
	\item[-] duur [1,2,3,5,8]: De duur representeert door een relatief getal om de geschatte tijd om de requirement te implementeren te representeren.
	      De waardes van de duur zijn een verkleinde selectie van Scrum poken \Parencite{ScrumPoker}.
\end{enumerate}
% De prioritering van de requirements wordt berekend door de volgende formule.
%
%
% De prioriteit van de requirements wordt berekend op basis van de invloed en het belang van de stakeholders,
% de tevredenheidsscore [1,2,\ldots,5] bij implementatie, de ontevredenheid score [1,2,\ldots,5] als het niet geïmplementeerd wordt en de duur van de implementatie (1,2,3,5,8) die wordt weer gegeven door een relatief getal.
% De berekening wordt gedaan door middel van de volgende formule:
% \begin{center}

\whitespace
	\begin{equation}
		\label{eq:prioritization}
		Score = SIP + TS - OS + (9 - duur)
	\end{equation}
% \end{center}

\whitespace
nadat er een score is berekend wordt er een prioriteit niveau toegegeven op basis van de MoSCoW methode:

\begin{itemize}
	\item{\makebox[3cm]{Must have:} $x \in \mathbb{R} : 14 \leq x \leq 18$}
	\item{\makebox[3cm]{Should have:} $x \in \mathbb{R} : 14 \leq x \leq 18$}
	\item{\makebox[3cm]{Could have:} $x \in \mathbb{R} : 14 \leq x \leq 18$}
	\item{\makebox[3cm]{Won't have:} $x \in \mathbb{R} : 14 \leq x \leq 18$}
\end{itemize}

\whitespace
De requirement worden vervolgens genoteerd in een gemodificeerde versie van een snow card.
Na het maken van de requirement wordt er terug gekoppeld met de stakeholder om het te verifiëren
als de volledige lijst gemaakt is wordt het gecheckt met de product owner en de bedrijfsbegeleider.
\whitespace
\todo[inline]{Maak fancy snow card en fancy formule en daarna tekst verbeteren scores kloppen ook nog niet want deze moeten besproken worden en eigelijk pas gesteld worden tijden de deelvraag ?}
\todo[inline]{Artifacts en formule}

