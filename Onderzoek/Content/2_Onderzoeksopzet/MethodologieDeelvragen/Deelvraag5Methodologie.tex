\subsection{Deelvraag 5: Prioritering}
Na het opstellen van een lijst met requirements als resultaat van deelvraag 4.
Deze lijst is echter nog niet bruikbaar, aangezien deze niet is geprioriteerd.
Om de prioriteiten van de requirements vast te stellen, wordt de volgende deelvraag geïntroduceerd.

\begin{center}
	\textit{\SubquestionFive}
\end{center}

\whitespace[0.2]
Dit wordt gedaan door middel van \textbf{requirements prioritization} zal er verschillende prioriteit niveaus toegekend worden aan de requirements.
Deze niveaus worden in beeld gebracht door middel van MoSCoW-methode \Parencite{MoSCoW}.
Om de prioritering te bepalen wordt er gebruik gemaakt van een formule (zie formule \ref{eq:prioritization}), in de ze formule wordt de volgende aspecten in mee genomen.
\begin{enumerate}
	\item[-] Tevredenheidsscore (TS) [1,2,\ldots,5]: dit is de waarde die door de stakeholder gegeven wordt als de requirement geïntroduceerd wordt. Waarbij een hoge waarde aangeeft als het belangrijk dat het geïntroduceerd wordt.
	\item[-] Ontevredenheidscore (OS) [1,2,\dots,5]: Dit is de waarde die door de stakeholder gegeven wordt wanneer het niet geïntroduceerd wordt. Waarbij de hoge waarde de ontevredenheid aangeeeft als het niet geïntroduceerd wordt.
	\item[-] Stakeholder invloed positie (SIP) [1,2,3,4]: Op basis van de stakeholders matrix wordt er een waarde aan een stakeholder groep toegekend 4 voor sleutelfiguren, 3 voor beinvloeder, 2 voor geïnteresseerde en 1 voor toeschouwer.
	\item[-] Duur [1,2,3,5,8]: De duur representeert door een relatief getal om de geschatte tijd om de requirement te implementeren te representeren.
	      De waardes van de duur zijn een verkleinde selectie van Scrum poker \Parencite{ScrumPoker}.
\end{enumerate}

\whitespace
\begin{equation}
	\label{eq:prioritization}
	Score = SIP + TS + OS + (9 - duur)
\end{equation}

\whitespace
Nadat er een score is berekend wordt er een prioriteit niveau toegegeven op basis van de MoSCoW methode.
De waardes van de prioriteiten zijn toegekend en gevalideerd door de product owner:

\whitespace
\makebox[3cm][l]{Must have:} $ x \in \mathbb{R} : 17 \leq x \leq 22 $ \\
\makebox[3cm][l]{Should have:} $ x \in \mathbb{R} : 12 \leq x \leq 16 $ \\
\makebox[3cm][l]{Could have:} $x \in \mathbb{R} : 7 \leq x \leq 11$ \\
\makebox[3cm][l]{Won't have:} $x \in \mathbb{R} : 4 \leq x \leq 6$

\whitespace
Als de requirements geprioriteerd zijn worden ze genoteerd in verschillende user stories.
Na het maken van de user stories wordt er terug gekoppeld naar de stakeholders om het resultaat te verifiëren.
Als de volledige lijst gemaakt is wordt de lijst gecheckt door de product owner en de bedrijfsbegeleider.

\todo[inline]{Scores zijn nu tijdelijk is nog geen moment om dit te valideren met de product owner en is ook pas handig dat gedaan wordt als de stories gemaakt zijn}
