\section{Elsa Croes en Eric Dijkstra interview}
\label{appendix:ExploreUserRequirementsElsa}

\subsection*{Vragen}
\begin{itemize}
    \item{Introductie}
    \begin{itemize}
        \item[1]{Wie ben je?}
        \item[2]{Wat doe je binnen Snakeware?}
    \end{itemize}
    \item{Applicatie}
    \begin{itemize}
        \item[3]{Wat zijn de kern functionaliteiten die de klant moet kunnen doen binnen het systeem?}
        \item[4]{Zijn er specifieke eisen en wensen die je graag in het proof of concept wilt zien?}
        \item[5]{Wat is jouw visie voor een nieuw CMS-systeem?}
        \item[6]{Wat is de prioriteit van SEO-integratie in het proof of concept?}
        \item[7]{Als de prioriteit hoog is hoe zie je dat dan voor je?}
        \item[8]{Wat is jouw visie van artikel types en formulieren binnen het proof of concept?}
    \end{itemize}
\end{itemize}


\subsection*{Transcriptie}


\DanteInt{Misschien even een klein voorstel rondje wie ben je en wat doen je binnen Snakeware?}

\ElsaInt{Ik ben Elsa Croes ben product owner binnen Snakeware vooral dus voor de kleinere klanten.}

\EricInt{Ik ben Eric Dijkstra ik ben frond end developer en val onder team 4 het team van elsa.}
\begin{center}
	\textit{Legt doel van het proof of concept uit}
\end{center}
\DanteInt{Wat zijn de kern functionaliteiten die jullie terug willen zien in het CMS?
	Wat zou de klant moeten kunnen doen in het CMS?}

\ElsaInt{Op het moment hebben we 2 soorten sites we hebben de content websites en de webshops.
	Ik denk dat het belangrijkste is dat de klant informatie kan presenteren op hun site op een mooie manier, afbeeldingen kunnen uploaden dat een beetje op een mooie manier.
}

\DanteInt{Dus snel en makkelijk content er op kunnen zetten.}

\DanteInt{Zijn er ook nog specifieke wensen die jullie willen zien in het CMS?}

\EricInt{Nou ja wat ik zelf vind waar we achter loopt met andere CMSen is het visuele gedeelte.
	Een poos geleden is daar een proof of concept voor gemaakt, maar daar is nooit verder op gebouwd.
	Maar dat mis ik nu wel erg.}
\begin{center}
	\textit{Legt Scope van het project uit}
\end{center}
\ElsaInt{Ik denk dat het belangrijkste is, want de basis staat er al, maar het door ontwikkelen.
	Als een klant nu zegt, weet ik veel een knopje ook rond kunnen maken dan geld dat nu voor alle klanten en dan moet het eerst naar R\&D en dan moet het daar besproken worden.
	Vinden wij default ja of nee dan gaat het door, eventueel of niet.
	Ik denk dat daar de grootste problemen nu eigenlijk liggen.}

\DanteInt{Dus het probleem is eigenlijk de flexibiliteit van hoe klanten nu content kunnen uploaden?
	Het liefst zou je deze stijling kunnen aanpassen in het CMS te kunnen doen.
}

\ElsaInt{Ja het mag meer flexibel zijn en als je nu iets aan past voor een site dan kunnen alle sites dat is nu een website die voor iedereen geld zeg maar.
	Het is natuurlijk wel custom voor de huisstijl natuurlijk, maar om iets extra te krijgen dan moet het ontwikkeld worden en dat geld dan voor alle klanten.}

\DanteInt{Dus ontwikkeltijd is het grootste probleem nu eigenlijk?}

\ElsaInt{Denk het wel}

\EricInt{Richting het CMS denk ik wel, het moet natuurlijk altijd goed getest worden.
	Dus ik snap wel waarom die duur zo lang is, maar het is natuurlijk niet gewild.}

\ElsaInt{Ik denk niet dat het maken van de functionaliteit maken en testen het meeste tijd kost.
	Maar ik denk dat het moment waar ik van de de vraag van de klant binnen krijg en de klant kan het gebruiken zit te veel tijd tussen.}

\DanteInt{Dus ik denk dat je dan meer kracht aan gebruiker of een expert gebruiker moet kunnen geven zodat dit soort kleine dingen niet door dit hele proces heen gaan.}

\ElsaInt{Ja het blijft een gezamelijk ding maar het blijft voor alle klanten}

\whitespace{Legt het project nog een keer uit.}

\ElsaInt{Nou ja de reden dat deze webstie mogelijkheden bestaat, is omdat de klant niet de “normale” doorloop tijd van een site betalen.
	Daarom zitten zij nu samen in een CMS met een concept.
	Normaal ben je met een knopje veranderen 2 uur bezig en dan kost het veel geld voor de klant.

	\whitespace
	Een klein bedrijf die heeft dat geld er niet voor en met dit websitemodel betalen ze \textit{tarief} per maand daarnaast hebben ze vrij weinig ontwikkelkosten.
	Dus ik denk dat het concept moet blijven dat ze in een CMS zitten, maar bijvoorbeeld dat het andere functionaliteiten heeft.
}

\DanteInt{Je hebt bijvoorbeeld Wix en Squarspace en daar heb je een fancy drag en drop website maker.}

\ElsaInt{
	Het mag inderdaad ook simplicitieser.
	Want wat je nu krijgt, is wat mensen moeten doen om een tekst te plaatsen is heel verwarrend en niet duidelijk wat je nou uit eindelijk moet doen.
	Als je bijvoorbeeld Eric pakt die weet exact wat hij moet doen en heeft het zo klaar.
	Maar nieuwe klanten er mee bezig gaan zijn ze altijd verward en weten ze niet exact wat ze moeten doen.
}

\EricInt{Het is niet heel intuïtief inderdaad.
	Je kan er niet heel makkelijk er door heen lopen als je het nog niet helemaal snapt.
	Ik denk dat als die visuele kant beter wordt weer gegeven dan zou dat ook veel oplossen aan het CMS-stukje dan.
	Als de klant kan zien dat de daadwerkelijke artikelen op de pagina terecht komen wordt het al veel duidelijker.
	Om nog een ander dingetje te noemen schoot me net naar binnen over het CMS.
	Ik vind het nu dat we heel erg gelimiteerd zijn aan de structuur van artikel types.
	Dus je hebt een vlak tekst veld een vlak tekst area en een datum veld en checkbox.
	Ik kan zelf nu niet exact plaatsen wat er dan nu bij zou moeten komen maar dat er ruimte is in layout van zo’n artikel.
}

\DanteInt{Ik heb van Rob Douma gehoord dat deze artikel types erg flexibel zijn.
	Ik weet niet of je dit exact bedoeld?}

\EricInt{Ja die bedoel ik.
	Dit is ook wat we altijd gebruiken bij de websites.
	Wij maken vaak zulke artikeltypes voor de klanten.
	Ik vind het soms apart om een aparte stijl te maken voor deze types me de radio buttons en zo.

	\whitespace
	Om je een voorbeeld te geven.
	Je hebt een optie voor een card en je hebt drie layouts voor die cards.
	En je maakt een keuze voor een andere layout, dan zou ik andere input velden voor deze optie verwachten.
	Wat er nu gebeurt, is dat alle velden te zien zijn voor alle layouts die dat type heeft wat het onnodig complex maakt.
	Wat ik graag zou willen is dat we dan alleen de velden zien die nodig zijn.
}

\ElsaInt{Het liefst zou je alleen de juiste keuzes tonen omdat je de klant dan helpt in de funnel}

\DanteInt{Dus als je een type hebt die meerdere opties heeft wil je alleen de opties zien die bij het geselecteerde type past.}

\ElsaInt{Ik denk dat de terminologieën ook wel mogen hernoemd worden.
	Ik heb soms nog steeds geen idee waar ze het soms over hebben.
	Je volgens mij een opmaak tabblad en dan moet kiezen kort of lang.
	En dan moet je exact weten welke categorie stijl er bij hoort.
	Nou het mag wel makkelijker voor mij want ik heb nu vaak het gevoel dat ik hoop dat ik het goede heb gedaan.

	\whitespace
	Die secties zijn ook niet flexibel want dat stellen wij een keer in voor de klant en dan kan de klant nooit meer veranderen.
	Buiten de footer en header om die moeten vast blijven staan, maar de rest moet gewoon flexibel zijn
}

\DanteInt{Je moet eigenlijk gewoon de content tussen de header en footer in moet je kunnen orderen hoe dat er uit ziet.}

\ElsaInt{Ja inderdaad, maar dat kan nu helaas niet altijd.}

\EricInt{We gebruiken nu wel veel van secties om functionaliteit terug te brengen in de site.}

\DanteInt{Dus eigenlijk wil je een manier hebben om functionaliteit te kunnen overerven op andere pagina’s}

\EricInt{Ja, het is belangrijk dat je die social media urls kan hergebruiken in de site.
	Als gebruiker zie je nu een lege sectie waar je niets mee kan en daar raken gebruiker nu vaak verward door.
	Eigenlijk wil je een optie dat je die boxes kan verstoppen zodat als je administrator deze nog wel kan zien maar als gebruiker niet.
}

\DanteInt{Ik had het met Rob Douma er ook over dat het groeperen en containeraizeren van items.
	Dat je bijvoorbeeld een groep kan gebruiken in een slider bijvoorbeeld zodat je makkelijk groepen van items kan gebruiken}

\EricInt{Ja zeker.
	Daar moeten we nu vaak een omweg voor doen om het goed voor elkaar te krijgen.
	Stel je wilt een rij met cards wil hebben moet ik nu custom logica maken om ze niet onder elkaar te tonen.
}

\DanteInt{Hebben jullie ook nog een specifieke eindvisie voor het CMS? Ver in de toekomst hoe zal het er dan uit zien?}

\EricInt{Ja een mooie drag en dorp editor met een standaard layout en dat je voorbeeld kleuren in hebt waar je gemakkelijk de stijl van de items kan aanpassen.
	Waarbij je op een paar knoppen klikt en dat je dan een hele andere layout hebt.
	En dat je dan placeholder dingen hebt voor inspiratie.
	Dan kijk ik weer terug op de WordPress omgeving waar ik veel in heb gewerkt.
	Dan heb je een thema en daar hoef je maar 3 keer door heen te klikken en dan heb je een hele andere website.}

\ElsaInt{Dan vind ik WordPress heel groot zeg maar. Dan denk ik een WordPress achtig concept maar dan simplistisch.
	Dat je niet 100 000 keuzes hebt, maar dat het super gestroomlijnd is wat je doet.}

\EricInt{Misschien de opties wat beter mee geven.}

\ElsaInt{het liefst willen we gewoon kunnen zeggen hey wil je website \textit{tarief} ja of nee, zo ja dan knallen we hem er in en dan is het klaar.}

\DanteInt{Snel en flexibel dus?}

\EricInt{Ja, als het nou paars of blauw is zou geen werk moeten kosten voor de frontenders.}

\ElsaInt{Het is natuurlijk een heel mooi product om winst op te draaien want dat moet natuurlijk ook gebeuren.
	Want je zou in theorie geen onderhoud aan specifieke klanten moeten hebben maar dat het een doorlopend product}

\DanteInt{Ik had Janny Reitsma ook laatst gesproken en die gaf aan dat SEO een erg hoge prioriteit heeft.
	Binnen Snakeware zelf ik weet niet hoe jullie dat zien?
	Hoe erg willen jullie SEO in het proof of concept terug komen.
}

\ElsaInt{Daar doe ik zelf eigenlijk niks mee en heb er persoonlijk geen verstand van, maar ik weet dat ze het beneden heel erg belangrijk vinden.}

\EricInt{Het is eigenlijk voor een kwalitatieve website heel belangrijk.
	Want anders word je niet gevonden op het internet en dan daalt de waarde van je website enorm want je hebt er minder aan.
	Je wordt niet gevonden op Google je wordt niet geïndexeerd. En als men je zoekt op een search engine dan komen ze niet tegen.}

\DanteInt{Dus het is dus heel belangrijk voor het proof of concept.}

\EricInt{Voor het proof of concept vind ik het wel een beetje anders want het is geen productieomgeving.
	Maar er moet zeker rekening mee gehouden moet worden.}

\DanteInt{De reden dat ik dit vraag is omdat in het oude CMS het na de tijd is toegevoegd en dat heeft impact gehad op andere design keuzes.}

\ElsaInt{Nee in dat geval is het wel belangrijk dat SEO in het begin wordt mee genomen.
	Dan moet de mensen beneden wel de opties er voor hebben.}

\DanteInt{Dus de prioriteit van SEO is dan hoog, laag of middel.}

\EricInt{Ik denk dat het net zo belangrijk is als de frontend van de site hoe het ingevuld moet worden.
	Dus ik zou zeggen erg hoog.}

\DanteInt{Hebben jullie ook een visie betreft het werken met artikel types en formulieren binnen het proof of concept.}

\ElsaInt{Een ding wat bij ons hoog op het lijstje staat is de webshop, maar dat is niet haalbaar voor het proof of concept natuurlijk.}

\EricInt{Ik weet niet of je de klant veel kracht wilt geven met de custom artikel types want dan kunnen ze het heel makkelijk een slechte website maken.
	Tenzij het een goede flow heeft waarbij dat niet kan.}

\ElsaInt{In dat geval is het ook belangrijk dat je de huisstijl kan locken voor een klant.
	Zodat ze er zelf geen prutsje van kunnen maken.
	En als ze een prutsje naar buiten brengen dan kunnen ze denken maakt Snakeware zulke sites dan hoeven wij er ook geen van snakeware.
	We willen wel hoog in de mark komen te staan natuurlijk.
	Maar de klant moet eigenlijk wel vrijheid hebben om de layout enzo aan te kunnen passen.
	Zodat het aan gegeven.}

\DanteInt{Dus de huisstijl integreren in het CMS is dus belangrijk.}

\ElsaInt{Yes, het is natuurlijk ook belangrijk dat de klant dit zelf zou kunnen aanpassen.
	En dan zou je natuurlijk met premissions werken wat het proof of concept natuurlijk ook moet doen.}

\DanteInt{Ik denk dat ik het meeste dan wel heb.
	Hebben jullie zelf nog dingen die je nog wilt behandelen?}

\EricInt{Ik vind het ook belangrijk dat de vormgeving direct goed werk op mobiel en desktop ect.
	Want dat is nu ook het geval.}

\ElsaInt{Over het algemeen werken wij mobile first en dat moet het ook natuurlijk goed op tablet en desktop werken.}


