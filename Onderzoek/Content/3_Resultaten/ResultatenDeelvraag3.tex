\section{Deelvraag 3: Knelpunten}
In dit hoofdstuk worden resultaat van deelvraag 3 \textit{\SubquestionThree} verzameld en geanalizeerd.
Voor deze deelvraag zijn er 2 expert interviews gedaan met Janny Reitsma en Rob Douma.
Het interview met Janny is uitgevoerd op 31 augustus 2023 en het interview met Rob op 2 november 2023.
In de volgende hoofdstukken de belangrijkste punten van de interviews genoemd en behandeld.
De interviews zijn opgenomen en zijn transcripties van gemaakt die te vinden zijn in bijlage \ref{appendix:ExpertInterviews}

\subsection{Janny Reitsma interview}
Janny Reitsma werkt voor de service desk van Snakeware, en geeft cursusen aan nieuwe klanten die het CMS gaan gebruiken.
Verder handeld Janny ook vaak de vragen en functionaliteit aanvragen van klanten af voor het CMS.

\whitespace
Tijdens het interview kwam het naar voren dat Janny vindt dat \gls{SEO} erg belangrijk is voor het CMS en dat het nu nog niet optimaal wordt afgehandeld.
Klanten kunnen op dit moment vaak fouten maken door verkeerd pagina's te linken aan elkaar waardoor het niet altijd goed gaat.
Verder moeten voor sommige \gls{SEO} opties elementen in de applicatie geplaatst worden.
Dit moet nu worden gedaan door een developer, dit kost tijd en geld hierdoor wordt Snakeware een duurdere partij.
Ditzelfde probleem speelt zich voor bij het implementeren van externe partijen.
Het liefst wilt Snakeware dat de klant dit zelf kan doen zodat Snakeware als goedkopere partij er uit komt.

\whitespace
Een andere groot punt is dat het huidige CMS interface niet meer van deze tijd is.
Het zou meer visueel moeten werken zodat klanten makkelijker kunnen zien waar hun content staat en hoe het getoont worden.

\subsection{Rob Douma interview}
Rob Douma is een oud informatie-analist die getransistioneerd is naar een product owner rol binnen Snakeware.
Verder richt hij vaak CMS omgeving in voor de grote klanten van het CMS en richt hij ook soms de database in.

\whitespace
Tijdens het interview kwam het naar voren dat Rob redelijk tevreden is met de huidige functionaliteiten is met het huidige CMS. 
Hierbij werd vooral gesproken over de formulieren functionaliteit en de aanpasbaarheid van de artikelen.
Volgens Rob moet zulke functionaliteit blijven zodat de klant meer zelf kan doen.

\whitespace
Een van de problemen die Rob wel aankaartte, is dat het CMS nog te rigide voelt in hoe content geplaatst moet worden.
Hierdoor kost het soms meer tijd om content goed op de site te kunnen zetten.
Een mogelijke oplossing hier voor gaf Rob aan is het groeperen van content op basis van een container model.
Waarbij je artikelen kan groeperen in een container en deze container kan injecteren in andere artikelen.

\whitespace
Tijdens het interview hebben we het ook over de huidige database structuur gehad.
Rob gaf aan dat er op dit moment veel legacy data in de database zit met alle gevolgen van dien.
Ook gaf Rob aan dat er veel logica in de database zit door middel van triggers en stored procedures, en dit is liever niet gewenst.

\subsection{Samenvatting en antwoord}
Er zijn twee interviewen gedaan met experts binnen Snakeware, deze experts waren Janny Reitsma en Rob Bouma.
In het interview van Janny kwam naar voren dat een van de grote knelpunten is de huidige implementatie van \gls{SEO}.
Ook kwam er naar voren dat de klant meer zelf moet kunnen inrichten door middel van derde partijen integratie.

\whitespace
Bij het interview van Rob kwam naar voren dat de huidige knelpunten bij de flexibiliteit van het inrichten van content.
Rob gaf ook aan dat hij graag wil dat de flexibiliteit van artikeltypes en formulieren moet blijven.
