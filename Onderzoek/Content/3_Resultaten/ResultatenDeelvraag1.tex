\section{Deelvraag 1: Stakeholders}
De stakeholders zijn individuen of organisaties die invloed of belang hebben bij het project.
De product owner zal de mogelijke markt van kleine klanten representeren.
Dit wordt gedaan omdat Snakeware niet kleine klanten heeft die gebruikt kunnen worden als stakeholders.
% Sommige externe stakeholders zullen gerepresenteerd worden door een gekwalificeerde interne medewerker van Snakeware.
% Dit wordt gedaan omdat de afstudeeropdracht een proof of concept is, en de klanten van Snakeware hier nog niet bij betrokken worden.
Als na de afstudeerperiode het een succes blijkt te zijn en Snakeware wilt het verder ontwikkelen dan wordt contact opgezocht met de externe stakeholders (potentiële kleinere klanten). 
Er is een invloed matrix gemaakt (figuur \ref{fig:StakeholdersInvloedMatrix}) om de invloed en belang van de stakeholders te visualiseren. 
Het project bestaat uit de volgende stakeholders:

\whitespace
\textbf{Hans Hoomans (CEO):}
Hans Hoomans is een van de oprichters van Snakeware en is verantwoordelijk (samen met de andere directieleden) voor de toekomstvisie van Snakeware.
Tijdens het opstellen van de opdracht is al aangegeven dat Hans veel ideeën heeft voor een nieuw \gls{CMS} als een product onder Snakeware.
Hierom is besloten om hem mee te nemen in het project om de toekomstvisie te integreren in het project.

\whitespace
\textbf{Product Owner:}
De Product Owner is verantwoordelijk voor het vertegenwoordigen van de belangen, eisen en wensen van de kleinere klanten.
Deze kleinere klanten worden niet als individuele stakeholders beschouwd, aangezien Snakeware geen afzonderlijke kleine klanten heeft.
Om deze reden wordt er binnen Snakeware een gekwalificeerde persoon ingezet om hen te vertegenwoordigen.

\whitespace
\textbf{Afdeling R\&D:} De afdeling R\&D van Snakeware zijn de ontwikkelaars van het huidige \gls{CMS} en kunnen veel inzicht bieden in de huidige situatie / problemen.
Tijdens de realisatie en ontwerpfase kan er advies gevraagd worden aan de backend en frontend developers van het R\&D team.
Na de afstudeerperiode wordt het project overgedragen aan het R\&D team.
%
% \whitespace
% Het product bestaat uit twee software-applicaties een frontend die de data weergeeft aan de gebruiker, en een \gls{CMS}-API die de data serveerd voor de frontend.
% Deze twee software-applicaties worden na de afstudeerperiode overgedragen aan twee verschillende diseplines in Snakeware, namelijk de backend en frontend developers.
%
% \whitespace
% \textbf{Backend developers:} De \gls{CMS}-API wordt aan het einde van de afstudeeropdracht overgedragen aan de backend developers van Snakeware.
% Tijdens de ontwerp en realisatie fase kan er voor advies gevraagd worden over hoe de \gls{CMS}-API het best ontworpen / gerealiseerd kan worden. \\
% \textbf{Frontend developers} : De interface applicatie die gemaakt wordt om de data te tonen aan de gebruiker wordt ook overgedragen aan het einde van de afstudeeropdracht.
% Tijdens de ontwerp en realisatie fase kan er voor advies gevraagd worden over hoe de inteface het best ontworpen / gerealiseerd kan worden.

\whitespace
\begin{graphic}
    \captionsetup{type=figure}
    \caption{Stakeholders invloed matrix}
    \includegraphics[scale=0.4]{StakeholdersInvloedMatrix}
    \label{fig:StakeholdersInvloedMatrix}
\end{graphic}

\todo[inline]{Het stakeholder verhaal uitzoeken na dat de herfst vakantie (voor of kleine klanten er we of niet er tussen moeten staan).}
\todo[inline]{De afbeelding klopt niet omdat hans er nog niet tussen staat en van wege het verhaal hier boven ik pas deze afbeelding aan als er bekend is wat er moet gebeuren met het verhaal hier boven.}
\todo[inline]{Net zoals ik zei in methodologie, als ik de product owner ben zou elsa dan de representatie zijn voor de kleine klant? Dit is vaag }
% \todo[inline]{Als er woorden over zijn maak een kleine samenvatting voor het resultaat}
