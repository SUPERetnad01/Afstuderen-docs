\section{Deelvraag 5: Prioritering}
\label{sec:Prioritering}
In dit hoofdstuk wordt de deelvraag beantwoord \textit{\SubquestionFive}
De requirement die uit deelvraag 4 zijn gekomen worden gecategoriseerd in 4 verschillende categorieën.
Deze categorieën zijn bepaald door middel van de MoSCoW methode \Parencite{MoSCoW}.
De MoSCoW methode bestaat uit de volgende onderdelen:

\whitespace
\textbf{Must have:} Dit zijn de kern functionaliteiten, zonder deze functionaliteiten zou het project niet bruikbaar zijn en niet als succes beschouwd worden.

\whitespace[1]
\textbf{Should have:} Dit zijn de functionaliteiten die niet essentieel zijn maar wel belangrijke functionaliteiten toevoegen in het systeem.

\whitespace[1]
\textbf{Could have:} Dit zijn de functionaliteiten die je graag zou willen hebben.
Ze zijn niet essentieel als er tijd over is in het project zou je deze functionaliteiten realiseren.

\whitespace[1]
\textbf{Won't have:} Dit zijn de functionaliteiten die je zou willen zien in een ander stadium van een project.
Of als er tijd over is en al de andere functionaliteiten zijn al geimplementeerd.
\whitespace[2]
Om de requirement te categoriseren in de categorieën van de MoSCoW methode wordt er een score aan de requirement toe gewezen.
Deze score wordt bepaald door middel van formule \ref{eq:PioritizationRequirementsFormula}.

\whitespace
\begin{equation}
	\label{eq:PioritizationRequirementsFormula}
	Score = TS + OS + (9 - duur)
\end{equation}

\whitespace
Formule \ref{eq:PioritizationRequirementsFormula} bestaat uit de volgende aspecten:
\begin{enumerate}
	\item[-] Tevredenheidsscore (TS) [1,2,\ldots,5]: Deze waarden wordt toegekend door de stakeholder met betrekking van de requirement.
	      De stakeholder geeft een waarde van 1 tot 5 van tevredenheid als het geimplementeerd wordt.
	      Hierbij is een 1 niet erg tevreden en 5 erg tevreden.
	\item[-] Ontevredenheidscore (OS) [1,2,\dots,5]: Deze waarden wordt toegekend door de stakeholder met betrekking van de requirement.
	      De stakeholder geeft een waarde van 1 tot van ontevredenheid als het niet geimplementeerd wordt.
	      Hierbij is een 1 niet erg onteverden en een 5 erg ontevreden.
	\item[-] Duur [1,2,3,5,8]: De duur representeert door een relatief getal om de geschatte tijd om de requirement te implementeren te representeren.
	      De waardes van de duur zijn een verkleinde selectie van Scrum poker \Parencite{ScrumPoker}.
	      Deze waardes worden geverifieerd door een senior developer zodat hier een correcte schatting van gemaakt kan worden.
\end{enumerate}

\whitespace
Het maximum wat doormiddel van deze formule behaald kan worden is 18 (5 + 5 + (9 - 1) = 18) en het minimum wat behaald kan worden is 3 (1 + 1 + (9 - 8) = 3).
Door het verschil van deze getalen te verdelen over 4 categorien krijg je de volgende getallen ranges:

\whitespace
\makebox[3cm][l]{Must have:} $ x \in \mathbb{R} : 14 \leq x \leq 18 $ \\
\makebox[3cm][l]{Should have:} $ x \in \mathbb{R} : 9 \leq x \leq 13 $ \\
\makebox[3cm][l]{Could have:} $ x \in \mathbb{R} :  5 \leq x \leq 8 $ \\
\makebox[3cm][l]{Won't have:} $ x \in \mathbb{R} : 3 \leq x \leq 4 $

\newpage
\whitespace
Het resultaat van formule \ref{eq:PioritizationRequirementsFormula} en de categorisatie met behulp van de MoSCoW methode is te zien in Tabel \ref{tab:RequirementPrioritization}. %\ref{}
De requirements zijn voorzien van een verwachte duur en acceptatiecriteria.
Een voorbeeld user story is te zien in tabel \ref{rq:res5Sample}.
Dit leidt tot 5 must have, 6 should have, 1 could have en 2 wont have requirements de volledige lijst met geprioriteerde requirements is te vinden in bijlage \ref{appendix:Requirements}.

\begin{graphic}
	\vspace{0.2cm}
	\captionsetup{type=table}
	\caption{gepriotiriseerde requirement}
	\begin{tabular}{ |p{3cm}||p{1cm}|p{1cm}|p{1.5cm}|p{1cm}|p{2.5cm}| }
		\hline
		\multicolumn{6}{|c|}{Requirement prioritizatie lijst}       \\
		\hline
		Requirement id & TS & OS & Duur      & Score & Prioritering \\
		\hline
		KB-FR1         & 5  & 5  & 9 - 5 = 4 & 14    & Must have    \\
		KB-FR2         & 5  & 5  & 9 - 5 = 4 & 14    & Must have    \\
		KB-FR3         & 4  & 1  & 9 - 3 = 6 & 11    & Should have  \\
		KB-FR4         & 3  & 1  & 9 - 2 = 7 & 11    & Should have  \\
		KB-FR5         & 4  & 3  & 9 - 3 = 6 & 13    & Should have  \\
		KB-FR6         & 5  & 5  & 9 - 5 = 4 & 14    & Must have    \\
		KB-FR7         & 3  & 2  & 9 - 5 = 4 & 9     & Should have  \\
		KB-FR8         & 5  & 5  & 9 - 3 = 6 & 16    & Must have    \\
		KB-FR9         & 3  & 2  & 9 - 3 = 6 & 11    & Should have  \\
		KB-FR10        & 2  & 1  & 9 - 5 = 4 & 7     & Could have   \\
		KB-FR11        & 4  & 2  & 9 - 3 = 6 & 12    & Should have  \\
		KB-FR12        & 5  & 5  & 9 - 5 = 4 & 14    & Must have    \\
		SW-FR13        & 5  & 5  & 9 - 3 = 6 & 16    & Must have    \\
		SW-FR14        & 1  & 1  & 9 - 8 = 1 & 3     & Wont have    \\
		SW-FR13        & 1  & 1  & 9 - 8 = 1 & 3     & Wont have    \\
		\hline
	\end{tabular}	\label{tab:RequirementPrioritization}
	\vspace{0.2cm}
\end{graphic}

\begin{table}[!ht]
	\caption{Requirement - KR-FR1}
	\label{rq:res5Sample}
	\begin{tabularx}{\textwidth}{|m{0.5cm}|l|m{2.0cm}|l|m{3.5cm}|l|}
		\hline
		\textbf{Id} & KB-FR1 & \textbf{Prioriteit} & Must have & \textbf{Verwachte duur} & 5                                                                                                         \\
		\hline
		\multicolumn{6}{|p{\dimexpr\linewidth-2\arrayrulewidth-2\tabcolsep}|}{\textbf{User Story}}                                                                                                   \\
		\hline
		\multicolumn{6}{|p{\dimexpr\linewidth-2\arrayrulewidth-2\tabcolsep}|}{Als klein bedrijf wil ik content kunnen plaatsen op mijn website, zodat ik mijn informatie kan delen op het internet.} \\
		\hline
		\hline
		\multicolumn{6}{|p{\dimexpr\linewidth-2\arrayrulewidth-2\tabcolsep}|}{\textbf{Acceptatiecriteria}}                                                                                           \\
		\hline
		\multicolumn{6}{|p{\dimexpr\linewidth-2\arrayrulewidth-2\tabcolsep}|}{
		\begin{itemize}
			\item{Als klein bedrijf moet ik een Content kunnen plaatsen op mijn site.}
			\item{Het moet mogelijk zijn om een artikel te plaatsen (een artikel is een stuk text met een titel).}
			\item{Het moet mogelijk zijn om een afbeelding te plaatsen.}
			\item{Het moet mogelijk zijn om een video te kunnen plaatsen.}
		\end{itemize}}                                                                                        \\
		\hline
	\end{tabularx}
\end{table}
