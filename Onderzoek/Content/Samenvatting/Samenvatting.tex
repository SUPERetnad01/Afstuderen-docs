\chapter*{Samenvatting}
Dit is het Onderzoeksverslag voor het \qw{Het CMS voor iedereen} project.
Het onderzoeksverslag is een onderdeel van de afstudeerperiode binnen NHL Stenden Hogeschool.
Het Onderzoek wordt uitgevoerd bij Snakeware New Media B.V. 
Snakeware is een E-business bureau gevestigd in Sneek, Nederland dat websites en andere digitale producten maakt voor klanten zoals DPG Media, DekaMarkt en Poiesz supermarkten.

\whitespace
Een van de producten die Snakeware aanbied is het Snakeware Cloud platform, een \gls{CMS} voor haar (grotere) klanten.
Door gebruik te maken van Snakeware Cloud kunnen klanten content plaatsen op hun site.
Snakware cloud is erg complex, dit komt door de verschillende functionaliteiten en configuratie mogelijkheden die het platform aanbiedt.
% Omdat Snakeware Cloud erg complex is te gebruiken en heeft veel configuratie stappen wat het lastig maakt te gebruiken voor onervaren gebruikers. 
Dit zorgt ervoor dat Snakeware niet relatief kleine webapplicaties kan leveren aan kleinere klanten vanwege de doorlooptijd en complexiteit van het systeem.
Hierdoor is het voor Snakeware niet mogelijk te vestigen in de markt van kleinere klanten.
Daarnaast is Snakeware Cloud aan de achterkant erg venouderd door de jaren heen, en kost het nu veel tijd om nieuwe functionaliteiten in te bouwen.

\whitespace
Om dit probleem op te lossen wilt Snakeware een nieuw proof of concept systeem voor een \gls{CMS} voor kleinere klanten.
Dit onderzoek wordt gedaan om de eisen en wensen van de stakeholders van dit systeem in beeld te brengen.
De lijst van requirements voor het systeem moeten voor 22 november 2023 gedefinieerd zijn.
Dit wordt gedaan door de volgende hoofdvraag te stellen:

\begin{center}
    \textit{\MainQuestion}
\end{center}

\whitespace
De hoofdvraag is beantwoord door de verschillende deelvragen te beantwoorden.
Voor de eerste deelvraag zijn de stakeholders in beeld gebracht.
Daarna is het systeem in beeld gebracht bij de tweede deelvraag.
De knelpunten van Snakeware Cloud zijn in beeld gebracht door de derde deelvraag.
Voor de vierde deelvraag zijn de eisen en wensen in beeld gebracht, deze eisen en wensen worden geprioriteerd in de vijfde deelvraag.

\whitespace
Door het beantwoorden van de hoofdvraag is er een lijst van geprioriteerde user stories opgesteld en randvoorwaarden voor het systeem.
Deze user stories worden gerealiseerd in de realisatiefase in de afstudeerperiode.
