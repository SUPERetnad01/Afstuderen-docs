\chapter{Inleiding}
Dit document is het onderzoeksverslag voor het \qw{Het CMS voor iedereen} project.
Dit is een onderdeel van het afstudeerperiode binnen NHL Stenden Hogeschool en word uit gevoerd bij Snakeware New Media B.V.
Dit onderzoek is een requirement analyse dat doelt om de eisen en wensen van de stakeholders in beeld te bregen en te vertalen naar requirements.
Het onderzoek omvat de \textit{requirement analyse} van de \gls{SDLC} (zie figuur \ref{fig:SDLC}).
Dit onderzoek zal gebruik maken van het DOT-Framework \Parencite{DOT-Framework} en \textit{Wat is onderzoek} \Parencite{Verhoeven}.   \\
\begin{graphic}
    \vspace{0.2cm}
    \captionsetup{type=figure}
    \caption{De Software development lifecyle afkomstig uit de afstudeer handleiding}
    \includegraphics[scale=0.2]{img/Placeholder}
    \label{fig:SDLC}
    \vspace{0.2cm}
\end{graphic}
In dit hoofdstuk wordt de organisatie omschreven daarna wordt de context, aanleiding en de afstudeeropdracht omschreven om een duidelijke context te schetsen van het systeem.
\section{Organisatieomschrijving}
Snakeware New Media B.V. (Snakeware) is een E-business bureau gevestigd in Nederland. Haar
aangeboden diensten omvatten het adviseren, bouwen en onderhouden van digitale producties, met
een focus op websites, webshops en mobiele apps (Snakeware, g.d.). Op het moment van schrijven
telt Snakeware meer dan 60 werknemers elk met verschillende specialiteiten. Ze leveren services
aan welbekende organisaties zoals DPG Media, DekaMarkt en Poeisz supermarkten.
\whitespace
Snakeware heeft een platform genaamd \qw{Snakeware Cloud} dit platform is een contentmanagementsysteem
(CMS) Waarmee ze digitale content kunnen leveren voor haar klanten.
Alle klanten die via Snakeware hun webapplicatie hebben maken gebruik van het Snakeware Cloud platform.
\whitespace
\todo[inline]{pictogram referentie maken van de bedrijfs structuur van Snakeware}


\section{Context}
Snakeware heeft een platform genaamd \qw{Snakeware Cloud} dit platform is een \gls{CMS} waarmee ze digitale content kunnen leveren voor haar (grotere) klanten.
Snakeware cloud is een applicatie waarmee Snakeware en haar klanten webapplicaties kan inrichten en voorzien van content.

\whitespace[2]
De klant van Snakeware kan zijn of haar website zelf inrichten door middel van het specificeren van de content op de verschillende pagina’s.
Dit wordt gedaan door middel van artikelen die door het \gls{CMS} gebruikt kunnen worden.
De content van het artikel kan verschillen tussen simpele tekst, vragenlijst, webshop items, etc.
Hiernaast zijn er ook \gls{SEO} opties binnen Snakeware cloud om de site goed te kunnen vinden op het internet.
Hierbij kun je denken aan titel tags en zoekwoorden kunnen toevoegen in de head \Parencite{HTMLhead}.

\whitespace[2]
Hierom heeft Snakeware cloud veel features en configuratie stappen wat het complex en duur maakt om een relatief kleine webapplicatie te maken voor kleinere klanten.
Dit zorgt ervoor dat Snakeware zich niet kan vestigen in een markt met veel kleinere klanten, en hierdoor omzet mis loopt.

\todo[inline]{\textit{Meer simepelere taal gebruiken?} weet niet of dat nodig is tho, maar snap je punt}
 % huidige en gewenste situatie
\section{Aanleiding} 
\todo[inline]{Deze text moet anders zijn dan de text in de probleem stelling. 
Voor nu niet super belangrijk maar aan het einde nog veranderen}
Het huidige platform is 21 jaar oud en er is veel functionaliteit in de loop der jaren aan toegevoegt.
Omdat Snakeware Cloud een oud platform is zijn er veel technieken en best practices gebruikt die nu niet meer als optimaal worden beschouwed.
Een voorbeeld hier van is tabel naam afkortingen achter elke kolom zetten, of hele grote files met alle functionaliteiten daar in. 
Ook zijn er technieken toegepast die nu niet meer relavant zijn, een voorbeeld hiervan is dat het \gls{CMS} gebruik maakt van javascript en toen ze er mee begonnen bestonden javascript classes nog niet dus hebben ze die zelf geimplenteerd.
Er zijn veel van dit soort gevallen wat het lastig maakt om Snakeware cloud te onderhouden of uit te breiden.
\whitespace
Een van de voornaamste kwesties met Snakeware Cloud betreft de verouderde datastructuur van de applicatie.
Deze veroudering is het gevolg van een initiële ontwikkeling waarbij onvoldoende rekening werd gehouden met toekomstige functionaliteitsuitbreidingen in het systeem.
Als gevolg daarvan is de onderliggende datastructuur niet aangepast, maar zijn er elementen aan toegevoegd.
Dit heeft geresulteerd in database queries van duizenden regels en complexe relaties tussen tabellen in de database.
it huidige scenario bemoeilijkt aanzienlijk het toevoegen van nieuwe functionaliteiten, wat resulteert in aanzienlijke tijds- en kosteninvesteringen.
\whitespace
Hierom wilt Snakeware dat er een nieuwe datastructuur komt met daar bij een \gls{CMS}-API.
Omdat er een nieuwe datastructuur moet komen en de logica van het oude systeem nauw verbonden is met de
datastructuur is het niet mogelijk om de oude code opnieuw te kunnen gebruiken.
% Dit platform moet een grote hoeveelheid potentiële kleine klanten kunnen ondersteunen naast de groteEr zijn veel van dit soort gevallen wat het lastig maakt om Snakeware cloud te onderhouden of uit te breiden.

\section{Opdrachtomschrijving}
De opdracht is om een proof of concept CMS API te ontwikkelen die gebruikt maakt van een ander datamodel en systeemarchitectuur.
Hierbij wordt ook een interface gemaakt waarbij de content getoond wordt voor de eindgebruiker.
Tijdens de afstudeeropdracht wordt er primair op het datamodel en de systeemarchitectuur gefocust.
Omdat er nog geen concreet datamodel en systeemarchitectuur is zal dit onderzocht/ontworpen moeten worden.
\whitespace
De opdracht omvat het achterhalen van de requirements, ontwerpen en ontwikkelen van het proof of
concept met als focus een nieuw datamodel, met de essentiële functionaliteiten. Omdat dit een proof of
concept project is, wordt er gebruikgemaakt van gekwalificeerde interne stakeholders die de wensen van
de klanten kunnen vertegenwoordigen om hier de functionele requirements uit op te halen.
\whitespace
De data van dit systeem wordt dan getoond op een front-end, zodat de eindgebruiker dan de content kan in
lezen en er mee kan interacteren. Er zal expliciet gefocust worden op de CMS API en de datalaag van dit
systeem.
\whitespace
\todo[inline]{Kan concreter, maar hoe?}
\whitespace
\todo[inline]{primair doel of doel?, het kan zijn dat er goede bevindingen uit komen voor grotere klanten ect. dit kan scope wel vergroten ect.}
Het primaire doel van het proof of concept is dat er aangetoond kan worden dat door het gebruiken van een nieuw
datamodel en nieuwe systeemarchitectuur ook services verleend kunnen worden aan kleinere klanten.
Dit zou evuntueel ook een start punt kan zijn om op verder van te bouwen.


% \section{Stakeholders}
\todo[inline]{examen comissie: 
\emph{Mits je sirieus onderzoek doet naar de requirements met je "gekwalificeerde interne stakeholders" zou 
het mogelijk moet zijn je opdracht te kaderen en concreet te maken wat er gemaakt moet gaan worden voor de PoC.
Toch denken wij dat eht verstandig is om ook een met de eindgebruikers te praten. maak de opdracht niet te groot en niet te klein}}
\todo[inline]{Wie zijn er betrokken bij dit project? Verwijzen naar bijlage?}
\subsection{Interne Stakeholders}
\subsection{Externe Stakeholders}
\subsection{Invloeds- en belangenmatrix}
\subsection{Relatie tussen stakeholders}

% \section{Leeswijzer}
\todo[inline]{Wat wordt er beschreven in dit document? In elk hoofdstuk?}
