\section{Opdrachtomschrijving}
De opdracht is om een proof of concept CMS-API te ontwikkelen die gebruikt maakt van een
datamodel en systeemarchitectuur dat flexibeler, onderhoudbaarder en moderner is. Hierbij
wordt ook een interface gemaakt waarbij de content getoond wordt voor de eindgebruiker.
Tijdens de afstudeeropdracht wordt er primair op het datamodel en de systeemarchitectuur
gefocust. Omdat er nog geen concreet datamodel en systeemarchitectuur is zal dit onder-
zocht/ontworpen moeten worden.

\whitespace
De opdracht omvat het achterhalen van de requirements, ontwerpen en ontwikkelen van
het proof of concept met als focus een nieuw datamodel, met de essenti¨ele functionaliteiten.
Omdat dit een proof of concept project is, wordt er gebruikgemaakt van gekwalificeerde
interne stakeholders die de wensen van de klanten kunnen vertegenwoordigen om hier de
functionele requirements uit op te halen.
Het huidige Snakeware cloud platform bestaat uit 2 verschillende interfaces: de Snakeware
cloud frontend en de webapplicatie. Met de Snakeware cloud frontend kan de klant de klant
de content van de website aanpassen. Door middel van de webapplicatie kan de eindgebrui-
ker de content bekijken en interacteren. Er is voor gekozen om niet de Snakeware cloud
interface te realiseren om de afstudeeropdracht in scope te houden.

\whitespace
Om de user journeys te testen wordt er gebruikgemaakt van postman workflows \Parencite{PostmanWorkflows} .
Het doel van het proof of concept is dat er aangetoond kan worden dat door het gebrui-
ken van een nieuw datamodel en systeemarchitectuur ook services verleend kunnen worden
aan kleinere klanten. Dit zou eventueel ook een startpunt kan zijn om op verder van te
bouwen.


% De opdracht is om een proof of concept \gls{CMS}-API te ontwikkelen die gebruikt maakt van een ander datamodel en systeemarchitectuur.
% Hierbij wordt ook een interface gemaakt waarbij de content getoond wordt voor de eindgebruiker.
% Tijdens de afstudeeropdracht wordt er primair op het datamodel en de systeemarchitectuur gefocust.
% Omdat er nog geen concreet datamodel en systeemarchitectuur is zal dit onderzocht/ontworpen moeten worden.
% \whitespace
% De opdracht omvat het achterhalen van de requirements, ontwerpen en ontwikkelen van het proof of
% concept met als focus een nieuw datamodel, met de essentiële functionaliteiten. Omdat dit een proof of
% concept project is, wordt er gebruikgemaakt van gekwalificeerde interne stakeholders die de wensen van
% de klanten kunnen vertegenwoordigen om hier de functionele requirements uit op te halen.
% \whitespace
% De data van dit systeem wordt dan getoond op een front-end, zodat de eindgebruiker dan de content kan in
% lezen en er mee kan interacteren. Er zal expliciet gefocust worden op de \gls{CMS} API en de datalaag van dit
% systeem.
% \whitespace
% % \todo[inline]{Kan concreter, maar hoe?}
% % \whitespace
% Het primaire doel van het proof of concept is dat er aangetoond kan worden dat door het gebruiken van een nieuw
% datamodel en nieuwe systeemarchitectuur ook services verleend kunnen worden aan kleinere klanten.
% Dit zou evuntueel ook een start punt kan zijn om op verder van te bouwen.
% % \todo[inline]{primair doel of doel?, het kan zijn dat er goede bevindingen uit komen voor grotere klanten ect. dit kan scope wel vergroten ect.}
