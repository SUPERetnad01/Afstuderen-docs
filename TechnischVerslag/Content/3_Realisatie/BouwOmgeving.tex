\section{Bouwomgeving}
Tijdens de laatste fase van het afstudeerproces is er verschillende software gebruikt.
Deze software omvat library's, frameworks, database tools en modelleringssoftware.
% Deze software verschilt zich tot library's, frameworks, database tools en modelleringssoftware.
In deze sectie wordt de software besproken en toegelicht waarom deze keuze gemaakt.

\whitespace[2]
\textit{Proces en documentatie}

\whitespace[2]
Voor het maken van documenten tijdens het afstudeerproces (Plan van aanpak, Onderzoek en Technisch verslag) is er gebruik gemaakt van \textbf{Latex} \parencite{Latex}.
De latex teksten zijn geschreven in de code-editor \textbf{Neovim} \parencite{NeoVim}.
Om de verschillende diagrammen te maken is er gebruik gemaakt van \textbf{Draw.io} \parencite{Drawio}.
Voor het noteren van het proces is er gebruik gemaakt van de notitie applicatie \textbf{Obsidian} \parencite{Obsidian}.
Hier werden de verschillende sprints bijgehouden en genoteerd.
Er is voor Obsidian gekozen omdat dit een simpele manier om de sprints bij te houden.
Binnen Snakeware wordt er gebruik gemaakt van het platform Jira \parencite{Jira}.
Jira is een erg extensief pakket wat voor de scope van de opdracht te groot is.

\whitespace[2]
\textit{Frontend}

\whitespace[2]
Voor het realiseren van de frontend applicatie is er gebruik gemaakt van \textbf{Vue 3} \parencite{Vue} en \textbf{Nuxt 3} \parencite{Nuxt}.
Vue 3 is een Javascript \parencite{JavaScript} / Typescript \parencite{Typescript} framework dat het makkelijker maakt om web applicaties te maken.
Nuxt 3 is een Vue framework wat het makkelijker maakt om Vue applicaties te maken en meerdere rendering opties aan biedt.
Er is voor Vue 3 met \textbf{Typescript} en Nuxt 3 gekozen om dat de gebruikte standaard is binnen Snakeware.
Hierdoor ontstaat er meer draagvlak binnen de frontend developers.
Voor het schrijven van de frontend code is gebruik gemaakt van \textbf{Webstorm} \parencite{Webstorm}.

\newpage

\whitespace[2]
\textit{Backend}

\whitespace[2]
Voor de backend code is er gebruik gemaakt van \textbf{C\# 12} \parencite{CSharp} en het \textbf{.NET 8} framework \parencite{DotNet8}.
Er is voor deze taal en framework gekozen omdat het een van de randvoorwaarden is van de afstudeeropdracht.
Verder wordt er gebruik gemaakt van \textbf{xUnit} \parencite{xUnit} om de unit test te maken.
Er is voor xUnit gekozen omdat dit framework het beste is in het isoleren van de testen \parencite{IsolationTest}. 
Verder is xUnit ook het meest gebruikte test-framework binnen snakeware.
Om de code te schrijven wordt er gebruik gemaakt van \textbf{Visual Studio 2022} \parencite{VisualStudio}.

\whitespace[2]
\textit{Overig}

\whitespace[2]
Voor het managen van de database is er gebruik gemaakt van \textbf{MongoDB Compas} \parencite{MongoDBCompas}.
Met deze tool is het mogelijk om querries uit te testen en de data makkelijk te bekijken.
Voor de frontend en backend is er gebruik gemaakt van versie beheer.
De tools die hier voor zijn gebruikt zijn \textbf{Git} \parencite{Git} en als repository platform \textbf{Bitbucket} \parencite{BitBucket}.
Verder is er voor het beheren van de docker omgeving gebruik gemaakt van \textbf{Docker desktop} \Parencite{DockerDesktop}.
