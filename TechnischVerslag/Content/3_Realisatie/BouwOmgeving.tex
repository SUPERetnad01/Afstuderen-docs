\section{Bouwomgeving}
Tijdens de laatste fase van het afstudeerproces is er verschillende software gebruikt.
Deze software verschilt zich tot library's, frameworks, database tools en modelleringssoftware.
In deze sectie wordt deze software besproken en toegelicht waarom deze keuze gemaakt.

\whitespace[2]
\textit{Proces en documentatie}

\whitespace[2]
Voor het maken van documenten tijdens het afstudeerproces (Plan van aanpak, Onderzoek en Technisch verslag) is er gebruik gemaakt van \textbf{Latex} \parencite{Latex}.
De latex teksten zijn geschreven in de code-editor \textbf{Neovim} \parencite{NeoVim}.
Om de verschillende diagrammen te maken is er gebruik gemaakt van \textbf{Draw.io} \parencite{Drawio}.
Voor het noteren van het proces is er gebruik gemaakt van het notitie applicatie \textbf{Obsidian} \parencite{Obsidian}.
Hier werden de verschillende sprints bijgehouden en genoteerd.

\whitespace[2]
\textit{Frontend}

\whitespace[2]
Voor het realiseren van de frontend applicatie is er voor gekozen voor \textbf{Vue 3} \parencite{Vue} en \textbf{Nuxt 3} \parencite{Nuxt}. 
Vue 3 is een \parencite{JavaScript} / Typescript \parencite{Typescript} framework dat het makkelijker maakt om web applicaties te maken.
Nuxt 3 is een Vue framework wat het makkelijker maakt om Vue applicaties te maken en meerdere rendering opties aan biedt.
Er is voor Vue 3 en Nuxt 3 gekozen om dat de gebruikte standaard is binnen Snakeware.
Hierdoor ontstaat er meer draag vlak binnen Snakeware binnen de frontend developers.
Voor het schrijven van de frontend code is gebruik gemaakt van \textbf{Webstorm} \parencite{Webstorm}.

\whitespace[2]
\textit{Backend}

\whitespace[2]
Voor de backend code is er gebruik gemaakt van \textbf{C\# 12} \parencite{CSharp} en het \textbf{.NET 8} framework \parencite{DotNet8}
Er is voor deze taal en framework gekozen omdat een  randvoorwaarden is dat het in de huistaal van Snakeware gerealiseerd moet worden.
Verder wordt er gebruik gemaakt van \textbf{xUnit} \parencite{xUnit} om de unit test te maken.
Om de code te schrijven wordt er gebruik gemaakt van \textbf{Visual Studio 2022} \parencite{VisualStudio}.

\whitespace[2]
\textit{Database}

\whitespace[2]
Voor het managen van de database is er gebruik gemaakt van \textbf{MongoDB Compas} \parencite{MongoDBCompas}.
Met deze tool is het mogelijk om querries uit te testen en de data makkelijk te bekijken.
%
% \whitespace[
% er is voor een nosql database gekozen (kijk naar ontwerp).
% voor de nosql database is er gekozen voor mongodb.
% mongodb is de grootste nosql database provider en heeft goede drivers voor c\#.
% \subsection{Tools}
Tijdens de ontwikkel en ontwerp fase is er gebruik gemaakt van schillende tooling.


Er zijn 2 verschillende applicaties ontwikkeld dit zijn de frontend en de backend applicatie.


% \subsection{Frameworks \& Libraries}
ddd

