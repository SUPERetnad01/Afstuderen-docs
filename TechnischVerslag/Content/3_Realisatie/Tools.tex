\section{Tools}
Er zijn 2 verschillende applicaties ontwikkeld dit zijn de frontend en de backend applicatie.

\whitespace[2]
\textit{Frontend}

\whitespace[2]
Voor de frontend is er gekozen om te werken met Vue en Nuxt 3. 
Vue is een Javascript / Typescript framework dat het makkelijker maakt om grotere frontenden te maken voor web applicaties.
Nuxt 3 is een Vue framework wat het makkelijker maakt om vue applicaties te maken en meerdere rendering opties aan bied.
Er is voor Vue en Nuxt 3 gekozen om dat de gebruikte standaard is binnen Snakeware.
Door dit te doen kunnen de frontend developers snel het proof of concept op pakken.
Voor het schrijven van de frontend code is gebruik gemaakt van \textbf{Neo Vim / webstorm}.

\whitespace[2]
\textit{Backend}

\whitespace[2]
Voor de Backend is gebruik gemaakt van C\# en .Net 8 een C\# framework.
Dit is gedaan omdat C\# de taal is waar alle andere services en developers gebuik van maken.
Er is voor .Net 8 gekozen omdat de laatste \textit{LTS} versie is van .Net.
De code voor de backend wordt geschreven in Visual Studio 2022.

\whitespace[2]
\textit{Database}
Er is voor een NoSQL database gekozen (kijk naar ontwerp).
Voor de NoSQL database is er gekozen voor MongoDB.
MongoDB is de grootste NoSQL database provider en heeft goede drivers voor C\#.
