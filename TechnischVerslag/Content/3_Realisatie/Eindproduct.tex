\section{Eindproduct}
Het doel van de afstudeer opdracht was een CMS systeem te maken.
Hierbij was de CMS interface weg te laten om de scope te beperken.
De gebruikte interface om met het CMS te interacteren is een swagger interface.
Swagger is een software oplossing dat de enpoints laat zien van een gegeven API.
Hierdoor is redelijk makkelijk om met het CMS te testen. 
% Omdat het eindproduct een proof of concept is er niet een traditionele interface voor de CMS-applicatie.
% De applicatie is bereiken door verschillende HTTP requesten te maken naar de API die opgesteld is.
% Voor een visuele inzage van endpoints is er gebruik gemaakt van Swagger wat standaard geïnstalleerd is bij .NET 8.
% De verschillende endpoints die gemaakt zijn te zien in figuur \ref{fig:SwaggerApi}.

\whitespace
\begin{graphic}
    \captionsetup{type=figure}
    \caption{Swagger interface}
    \includegraphics[scale=0.2]{Placeholder.jpg}
    \label{fig:SwaggerApi}
\end{graphic}

\whitespace
\textbf{Datamodel implementatie}

\whitespace
Om het datamodel wat beschreven staat in paragraaf \ref{subsection:Datamodel}.
In dit datamodel wordt er gebruik gemaakt van een Content datatype wat meerdere verschillende types kan zijn.
Dit is makkelijk op te lossen in niet statische getypte talen, maar in statische talen is dit een kleine uitdaging.

\whitespace
Dit is opgelost door gebruik te maken van een abstracte superklassen \textit{FieldValueBase}.
Deze superklassen heeft verschillende implementaties die de verschillende datatypes ondersteunen.
Een klasse diagram hier van is te zien in figuur \ref{fig:ImplementatieFields}.

\whitespace
\begin{graphic}
    \captionsetup{type=figure}
    \caption{Swagger interface}
    \includegraphics[scale=0.2]{Placeholder.jpg}
    \label{fig:ImplementatieFields}
\end{graphic}

Wat misschien kan opvallen is dat er gebruikt wordt van een dictionary wat een \textit{Language} als key heeft.
Dit is gedaan zodat elke field een optie voor multilingual te implementeren.
Er zijn ook varianten die maar waarde opslaan om het zo generiek mogelijk te houden.

\whitespace
Het nadeel van abstracte klasses zijn dat de standaard C\# \textit{Json Deserialisation} oplossing niet werkt met het interpeteren van de data.
Om dit probleem op te lossen is er gebruik gemaakt van ModelBinders.
ModelBinders zijn een C\# oplossing waarbij je logica kan uitvoeren voordat de data bij de controller aan komt.
Hier wordt gekeken naar welke implementatie van de \textit{FieldValueBase} nodig is en wordt dan vervolgens correct gesiraliseerd.

