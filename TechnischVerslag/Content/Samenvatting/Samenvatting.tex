\chapter*{Samenvatting}
Dit document is het technisch verslag voor het afstudeertraject van Dante Klijn.
Het technisch verslag is een van de onderdelen voor het afstudeertraject bij NHL-Stenden hogeschool.
De afstudeeropdracht is uitgevoerd bij Snakeware, een E-business bureau gevestigd in Sneek.

\whitespace
Snakeware heeft een \gls{CMS} platform waarmee ze digitale content kunnen voorzien voor haar (grotere) klanten.
Het huidige \gls{CMS} is 21 jaar oud gebruikt veel oude technieken en houd zich niet meer aan de huidige code standaarden binnen Snakeware.
Snakeware ziet potentie om een nieuw \gls{CMS}-platform te maken en gebruik te maken van nieuwe technieken en een nieuw datamodel.
Hierdoor zou mogelijk ook services verleend kunnen worden aan kleinere klanten.
De realisatie van dit product is gedocumenteerd in dit verslag.

\whitespace
Voordat dit verslag is gemaakt is er een onderzoek uitgevoerd om de verschillende eisen en wensen in beeld te krijgen.
Hieruit is een lijst met geprioriteerde requirments gekomen die gebruikt zijn om het eindproduct te realiseren.

\whitespace
Op basis van de requirments is er een ontwerp gemaakt.
Dit ontwerp omvat het nieuwe datamodel en een nieuwe softwarearchitectuur.
Voor het ontwerpen is er gebruik gemaakt van het 4 + 1 view model.
Hiervoor zijn er verschillende UML-diagrammen gemaakt om het ontwerp in beeld te krijgen.
Verder is er ook een test plan opgesteld doormiddel van het V-Model.

\whitespace
Met het ontwerp is het proof of concept \gls{CMS} gerealiseerd.
Dit systeem kan voor doormiddel van het datamodel simpele en complexe content renderen op de \gls{Eindgebruiker} webapplicatie.
Het systeem voldoet bijna aan de verschillende eisen en wensen die voor het proof of concept zijn gesteld. 
Er is tijdens het realiseren is er besloten om twee van deze requirments niet te realiseren.

\whitespace
Er wordt Snakeware aanbevolen om het proof of concept verder door te ontwikkelen.
De meest invloedrijke uitbreidingen zouden zijn het toevoegen van een interface voor de \gls{Beheerder}.
Als dat gerealiseerd is zou de vervolg stap zijn om webshop optie toe te voegen.
