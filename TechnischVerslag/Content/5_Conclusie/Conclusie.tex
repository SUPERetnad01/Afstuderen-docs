\chapter{Conclusie}
Het doel van de afstudeeropdracht was het maken van een proof of concept \gls{CMS} systeem te maken.
Dit product zou moeten aantonen dat door het gebruik te maken van een nieuw datamodel en systeemarchitectuur.
Door gebruik te maken van een flexibel datamodel zou er minder maat werk gedaan  moeten worden waardoor er ook services verleend kunnen worden aan kleinere klanten.
Tijdens de afstudeer periode is de \gls{SDLC} een keer doorlopen.

\whitespace
\begin{center}
\textit{Requirement analysis fase}
\end{center}

\whitespace
Tijdens deze fase is er een onderzoek gedaan naar de requirement van het proof of concept applicatie \Parencite{DanteOnderzoek}.
Uit dit onderzoek is een lijst van geprioriteerde requirements uit gekomen die gebruikt zijn om het product te realiseren.

\whitespace
\begin{center}
\textit{Designfase}
\end{center}

\whitespace
Tijdens de ontwerp fase is er een datamodel en systeemarchitectuur die de requirements kan ondersteunen.
Hierbij is een concreete manier van het opbouwen van de software in beeld gebracht (zie sectie \ref{subsecion:SoftwareArhitectuur}).
Verder is er een concreet datamodel ontworpen dat complexe structuren van content kan bevatten zonder extra code toe te voegen.

\todo[inline]{Niet super tevreden mee}

\whitespace
\begin{center}
\textit{Implementatiefase}
\end{center}

\whitespace
Na dat het ontwerp gemaakt was is het geimplementeerd in 2 producten, een Nuxt 3 frontend applicatie en een .NET 8 API.
Voor het realiseren van de API is de gekozen softwarearchitectuur geimplementeerd.
Tijdens het realiseren van het eindproduct is er gebruik gemaakt van de agile methode Scrum.
Er werdt elke werkdag een  \textit{Daily standup} gedaan met de leden van het R\&D team. 

\whitespace
\begin{center}
\textit{Testfase}
\end{center}

\whitespace
De testfase liep tegelijk met de implementatiefase van het project.
Voor het maken van de testen is er een testplan gemaakt die tijdens de testfase doorlopen is.
Het testplan maakt gebruik van het V-Model waarbij elke stap doorlopen is behalve de acceptatie testen.
Dit is gedaan om dat de functionaliteiten moeilijk te testen zijn door onbekende gebruikers.
Voor het unit testen van de API is er gebruik gemaakt van xUnit.
Dit is gedaan omdat het beste geisoleerde testen kan runnen vergeleken met de andere frameworks.

\whitespace
Het proof of concept voldoet bijna aan alle eisen en wensen van de stakeholders op 2 na.
Dit zijn het aan kunnen maken van formulieren en het authenticeren van \gls{Beheerder}s.
Voor het kunnen authenticeren en indentificeren van beheerders is tijdens de realisatiefase gerealiseerd dat de prioriteit van de requirement verkeerd ingeschat was.
Daarom is besloten om de requirement niet te realiseren vanwege dat er andere requirements een hogere prioriteit hadden.

\whitespace
Het kunnen aan maken van formulieren en het gebruik daar van bleek een te grote functionaliteit te zijn voor de realisatiefase.
Hierom was besloten omdit in een mogelijk volgend stadium te implementeren.

