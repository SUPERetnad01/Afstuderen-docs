\subsection{Datamodel}
Een van de belangrijkste doelen van de afstudeeropdracht is om een nieuw datamodel te maken voor een CMS-systeem.
Dit datamodel moet veel verschillende soorten datastructuren kunnen ondersteunen en deze kunnen presenteren op een klant zijn website.
Het huidige CMS van Snakeware doet dit doormiddel van een complexe structuur.
Deze structuur zorgt ervoor dat het moeilijk is om nieuwe functionaliteit toe te voegen en dat het systeem lastig is te onderhouden.

\whitespace
Daarom heeft het nieuwe datamodel twee belangrijke uitgangspunten om de pijnpunten van het oude CMS te voorkomen.
Het datamodel moet generiek blijven, zodat er veel verschillende datastructuren in opgeslagen kunnen worden.
Door het datamodel generiek en flexibel te houden hoeft het systeem niet uitgebreid te worden om een nieuwe datastructuren te ondersteunen.
Verder moet de structuur van het datamodel simpel blijven zodat het makkelijk te onderhouden is.
Het ontwerp bestaat uit verschillende onderdelen met verschillende taken.
Er wordt op elk onderdeel ingezoomd om een beter beeld te schetsen aan het ontwerp.

\whitespace[2]
\textbf{De content}

\whitespace
Om de content op te slaan wordt er gebruikgemaakt van een geneste structuur.
Om deze geneste structuur op te bouwen is er gekozen om de data op te delen in 2 groepen.

\whitespace
\begin{itemize}
    \item[1] \textbf{Simpele data}: hier worden de basis types van het systeem in opgeslagen.
    Dit zijn de primaire types van de content voorbeelden van deze types zijn tekst, nummer en boolean.
    In het ontwerp is simpele data genoteerd als \textbf{fieldvalues}.
    \item[2] \textbf{Complexe data} is een samenstelling van 0 of meerdere \textbf{fieldvalues}.
    Verder kan complexe data meerdere complexe data bevatten. 
    Hierdoor kunnen er complexe structuren gemaakt worden.
    Dit is gerepresenteerd als \textbf{ItemValues} in het ontwerp.
\end{itemize}

\whitespace
Om structuur aan de data te geven worden ze voorzien door een definitie.
Er zijn definities voor \textbf{itemvalues} en \textbf{fieldvalues}.
Dit is gedaan zodat als er een nieuwe instantie van een stuk complexe data gemaakt moet worden dat dit op basis van de definitie gedaan kan worden.
Om dit weer te geven is er een klassendiagram gemaakt die te zien is in figuur \ref{fig:DeContentClassDiagram}.

\whitespace[2]
\begin{graphic}
    \captionsetup{type=figure}
    \caption{Klassendiagram ItemValue}
    \includegraphics[scale=0.35]{ItemValueEntityClassDiagram.png}
    \label{fig:DeContentClassDiagram}
\end{graphic}

\newpage

\whitespace
\textbf{De componenten}

\whitespace
Om de content een vorm te geven wordt er gebruik gemaakt van componenten.
Voorbeelden van componenten zijn card, artikel en afbeelding.
Door de componenten en de content te scheiden van elkaar geeft dat de mogelijkheid om verschillende componenten te gebruiken op hetzelfde stuk content.
Om ervoor te zorgen dat een component een stuk content kan interperteren wordt er gebruik gemaakt van definities.
Als een stuk content alle \textit{required FiedlDefinitions} heeft kan de component de content renderen.
De componenten worden gerepresenteerd door \textbf{visualcomponent} in het ontwerp. 
Een klasse diagram van de visual component is te zien in figuur \ref{fig:VisualComponentEntityClassDiagram}.

\whitespace[2]
\begin{graphic}
    \captionsetup{type=figure}
    \caption{Klassendiagram VisualComponent}
    \includegraphics[scale=0.8]{VisualComponentEntityClassDiagram.png}
    \label{fig:VisualComponentEntityClassDiagram}
\end{graphic}

\whitespace
\textbf{De visualalisatie}

\whitespace
Om content op een pagina te renderen moet de content gekoppeld worden samen met de componenten.
Dit wordt gerepresenteerd door het \textbf{itemvisual} in het ontwerp.
Verder kan een itemvisual ook meerdere itemvisuals bevatten waardoor er een geneste structuur ontstaat.
Hierdoor is het mogelijk om complexe websitestructuren te maken.
Verder wordt ook op het itemvisual stijling en layout meegegeven zodat dit niet component afhankelijk is.
Het klassen diagram voor het gehele ontwerp is te zien in figuur \ref{fig:ClassDiagramEntireSystem}

\newpage
\begin{graphic}
    \captionsetup{type=figure}
    \caption{Klassendiagram ItemValue}
    \includegraphics[scale=0.45,angle=90]{ClassDiagramEntireSystem.png}
    \label{fig:ClassDiagramEntireSystem}
\end{graphic}

% \whitespace
% Het datamodel bestaat uit 4 \textbf{objecten} wat in zijn geheel leidt naar de uiteindelijke data die naar de frontend wordt gestuurd.
% Deze objecten zijn Item Definition, Item Value, Visual Component en ItemVisual.
%
% \whitespace[2]
% \textbf{Item value}: De content van het CMS wordt opgeslagen in het itemvalue entiteit.
% Waarbij de waardes van de content worden opgeslagen in een of meerdere \textbf{fieldvalues}.
% Deze fieldvalues kunnen verschillende data types hebben zoals string, boolean en interger.
% Verder kunnen itemvalues meerdere itemvalues bevatten waardoor er complexe datastructuren kunnen ontstaan.
% Om meer inzicht in het datamodel te geven is er een klassendiagram van de itemvalues in figuur \ref{fig:ItemValueEntityClassDiagram}.
%
% \whitespace[2]

%
% \whitespace[2]
% \textbf{Item Definition}: Om structuur aan item values te geven wordt er gebruikgemaakt van een item definition (zie figuur \ref{fig:ItemDefinitionClassDiagram}).
% De belangrijkste functionaliteit van de definition is om aan te geven welke velden er op verschillende items zitten en welke daarvan verplicht zijn.
%
% \whitespace[2]
% \begin{graphic}
%     \captionsetup{type=figure}
%     \caption{Klassendiagram  ItemDefinition}
%     \includegraphics[scale=0.5]{ItemDefinitionClassDiagram.png}
%     \label{fig:ItemDefinitionClassDiagram}
% \end{graphic}
%
% \whitespace
% \textbf{VisualComponent}: Om de data te renderen moeten er componenten gebruikt worden in de frontend om dit af te handelen waar nodig.
% De visualcomponent wordt gebruikt om deze componenten aan te geven welke er zijn en welke definition er bij hoort.
% Een klasse diagram van de visual component is te zien in figuur \ref{fig:VisualComponentEntityClassDiagram}.
%

%
% \whitespace[2]
% \textbf{ItemVisual}: Dit is het object dat de visualcomponent en de item value samen voegt tot een geheel waardoor er content gerenderd kan worden op de pagina.
% Verder geeft dit object ook aan welke mogelijke stijling of layout op het item moet worden toegepast (zie figuur \ref{fig:ItemVisualEntityClassDiagram}.
% Om de data te renderen op een frontend wordt de itemVisual geparesed naar een itemvisualDTO zie figuur \ref{fig:ItemVisualDTOClassDiagram}.
%
% \whitespace[2]
% \begin{graphic}
% 	\captionsetup{type=figure}
% 	\caption{Klassendiagram ItemVisual}
% 	\includegraphics[scale=0.6]{ItemVisualEntityClassDiagram.png}
% 	\label{fig:ItemVisualEntityClassDiagram}
% \end{graphic}
%
% \whitespace[2]
% \begin{graphic}
% 	\captionsetup{type=figure}
% 	\caption{Klassendiagram ItemVisualDTO}
% 	\includegraphics[scale=0.4]{ItemVisualDTO.png}
% 	\label{fig:ItemVisualDTOClassDiagram}
% \end{graphic}
