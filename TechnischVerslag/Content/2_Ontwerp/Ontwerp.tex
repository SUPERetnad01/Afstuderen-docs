\chapter{Ontwerp}
In dit hoofdstuk wordt het ontwerp van de softwareproducten beschreven.
Tijdens de eerste fase van de afstudeerperiode is er een onderzoek gedaan naar de requirements van het systeem \parencite{DanteOnderzoek}.
Het resultaat van dit onderzoek is een lijst van geprioriteerde requirements en randvoorwaarden waar het systeem aan moet voldoen.
Deze resultaten zijn gebruikt om het softwareproduct te ontwerpen.

\whitespace
Voor het ontwerpen van het systeem is er gebruikgemaakt van een ontwerp framework genaamd het 4 + 1 view model \parencite{4+1ViewModelPaper}.
Het 4 + 1 view model maakt gebruik van 5 verschillende perspectieven om de software inbeeldt te krijgen.
Deze perspectieven zijn scenarios, logical, process, development en physical view (visualisatie is te zien in figuur \ref{fig:4p1Model}).
In de volgende secties worden de perspectieven uitgelegd en ingevuld. 

\whitespace[2]
\begin{graphic}
	\captionsetup{type=figure}
    \caption{4 + 1 Model view model \parencite{4+1ViewModelPaper}}
	\includegraphics[scale=1.4]{4+1ModelView.png}
	\label{fig:4p1Model}
\end{graphic}

\newpage
\section{Scenario's}
De scenario view is een representatie van de belangrijkste use cases van het systeem \parencite{4+1ViewModelPaper}.
De use cases zijn opgesteld door middel van de geprioriteerde lijst van requirements van het onderzoek \parencite{DanteOnderzoek}.
Om de verschillende use cases en interactie met andere actoren in het systeem in beeld te krijgen wordt er gebruik gemaakt van een use case diagram \parencite{UseCaseDiagram}.
In figuur \ref{fig:UseCaseDiagramCMS(1)}, \ref{fig:UseCaseDiagramCMS(2)} en \ref{fig:UseCaseDiagramBekijkenVanContent} zijn de verschillende use case diagramen te zien.
Er is voor gekozen om de use cases voor het CMS-platform op te delen om het meer overzichtelijk te maken.

\whitespace[2]
\begin{graphic}
	\captionsetup{type=figure}
	\caption{Use case diagram CMS(1)}
	\includegraphics[scale=0.45]{UseCaseDiagramCMS(1).png}
	\label{fig:UseCaseDiagramCMS(1)}
\end{graphic}

\whitespace[2]
In figuur \ref{fig:UseCaseDiagramCMS(1)} zijn de verschillende use cases te zien die te maken hebben met de content binnen het CMS.
Voor de overige functionaliteiten is er in figuur \ref{fig:UseCaseDiagramCMS(2)} ook een use case diagram te vinden.
Dit use case diagram toont het authenticatie en algemene site functionaliteit.
Het laatste diagram (figuur \ref{fig:UseCaseDiagramBekijkenVanContent}) is gemaakt voor de klanten site.
Dit is de locatie waar de (website) bezoeker de content kan bekijken.

\newpage
\begin{graphic}
	\captionsetup{type=figure}
	\caption{Use case diagram CMS(2)}
	\includegraphics[scale=0.4]{UseCaseDiagramCMS(2).png}
	\label{fig:UseCaseDiagramCMS(2)}
\end{graphic}

\whitespace
\begin{graphic}
	\captionsetup{type=figure}
	\caption{Use case diagram klant site}
	\includegraphics[scale=0.4]{UseCaseDiagramBekijkenVanContent.png}
	\label{fig:UseCaseDiagramBekijkenVanContent}
\end{graphic}

\newpage
\section{Logical View}
In de volgende sectie wordt de logical view van het 4 + 1 model toegelicht.
Het doel van de logical view is de functionaliteiten van het systeem in beeld te brengen \parencite{4p1Model}.
Dit wordt gedaan door op een abstract niveau naar de structuur en datamodel van het systeem te kijken zonder implementatie details.
De volgende sub secties gaan dieper in op het datamodel en de softwarearchitectuur van het systeem.

\subsection{Datamodel}
Het Datamodel bestaat uit 4 \textbf{objecten} wat in zijn geheel leid naar de uitendelijke data die naar de frontend wordt gestuurd.
Deze objecten zijn Item Definition, Item Value, Visual Component en ItemVisual.

\whitespace[2]
\textbf{Item Value}: De Item Value slaat encapsuleert de content/data van het systeem op.
Item Values kunnen meerdere Item values bevatten, waardoor je een geneste structuur krijgt.
De Item value bevat ook 1 of meerdere FieldValueEntityBase.
Deze abstracte class zorgt er voor dat er verschillende data types kunnen opgeslagen worden in het het zelfde item.
In het klassen diagram (zie figuur \ref{fig:ItemValueEntityClassDiagram}) zijn de huidige implementatie mogelijkheden.
Deze zijn String,Bool,Int en Api deze lijst zou in de toekomst uitgebruikt kunnen worden.
Een bezondere is de Api FieldType, met dit veld maakt je een http get request naar een ander url.
Hierdoor kan je dynamisch externe content ophalen, bijvoorbeelde de Oembed data van een youtube video.

\whitespace[2]
\begin{graphic}
    \captionsetup{type=figure}
    \caption{klassen diagram ItemValue}
    \includegraphics[scale=0.7]{ItemValueEntityClassDiagram.png}
    \label{fig:ItemValueEntityClassDiagram}
\end{graphic}

\whitespace[2]
\textbf{Item Definition}: Om structuur te geven aan de Item values wordt er gebruik gemaakt van een item Definition.
        De belangrijkste functionaliteit van de definition is om aan te geven welke velden er op verschillende items zitten en welke daarvan verplicht zijn.

\whitespace[2]
\begin{graphic}
    \captionsetup{type=figure}
    \caption{klassen diagram ItemDefinition}
    \includegraphics[scale=0.8]{ItemDefinitionClassDiagram.png}
    \label{fig:ItemDefinitionClassDiagram}
\end{graphic}

\whitespace[2]
\textbf{VisualComponent}: Om de data te renderen moeten er components gebruikt worden in de frontend om dit af te handelen waar nodig.
De VisualComponent wordt gebruikt om deze componenten aan te geven welke er zijn en welke definition er bij hoordt.

\whitespace[2]
\begin{graphic}
    \captionsetup{type=figure}
    \caption{Klassen diagram VisualComponent}
    \includegraphics[scale=0.8]{VisualComponentEntityClassDiagram.png}
    \label{fig:VisualComponentEntityClassDiagram}
\end{graphic}

\whitespace[2]
\textbf{ItemVisual}: Dit is het object dat de VisualComponent en de Item Value samen voegt tot een geheel waardoor er content gerenderd kan worden op de pagina.
Verder geeft dit object ook aan welke mogelijke stijling of layout op het item moet worden toegepast.

\whitespace[2]
\begin{graphic}
	\captionsetup{type=figure}
	\caption{Klassen diagram ItemVisual}
	\includegraphics[scale=0.8]{ItemVisualEntityClassDiagram.png}
	\label{fig:ItemVisualEntityClassDiagram}
\end{graphic}

\whitespace[2]
Om de data te renderen op een froentend wordt er gebruik van de ItemVisualDTO zie figuur \ref{fig:ItemVisualDTOClassDiagram}.
De data wordt hier genest en geordend op basis van de ordering.

\whitespace[2]
\begin{graphic}
	\captionsetup{type=figure}
	\caption{Klassen diagram ItemVisual}
	\includegraphics[scale=0.7]{ItemVisualDTO.png}
	\label{fig:ItemVisualDTOClassDiagram}
\end{graphic}

\newpage
\subsection{Software architectuur}
Om de onderhoudbaarheid te verbeteren van het huidige CMS is er ook nagedacht over het opzetten van de code.
Hiervoor is er gekeken hoe de architectuur zich houd aan de verschillende SOLID principes \parencite{SOLID}. 
SOLID is een acroniem dat 5 verschillende principes in houd.
Deze principes zorgen er voor dat de code beter te onderhoudbaar is en makkelijker te begrijpen is.
SOLID bestaat uit de volgende principes:

\begin{enumerate}
    \item \textbf{Single-responsibility principle}
    Dit betekent dat een class, maar 1 verantwoordelijk mag hebben en bij definitie ook maar 1 taak.
    Als een class te veel verantwoordelijkheden heeft dan wordt het moeilijker om de code te begrijpen en aan te passen.

    \item \textbf{Open-closed principle}
    Dit houdt in als class of functie of een andere software entiteit uitgebreid moet worden dat het wordt gedaan door middel van een extentie(open) in plaats van modificatie(closed).
    Hierdoor hou je oude code intact en heb je geen risico dat je bestaande code stuk gaat vanwege de nieuwe functionaliteit die geschreven is. 
        
    \item \textbf{Liskov substitution principle} 
    Een class die afgeleid is van een base class, zou moeten vervangen kunnen worden voor  een andere instantie van die base class.
    Dit moet gedaan kunnen worden zonder de validiteit (corectness) van het programma te beïnvloeden.
    Door het gebruik van het Liskov substitution principle verhoog je de consistentiteit en de verwachte uitkomst van je progamma.

    \item \textbf{Interface segregation} 
    Een interface moet alleen de methods geven die nodig is voor de client. 
    Geen client moet geforceerd zijn om methodes te implementeren waar die geen gebruik van maakt.
    Dit uit zich vaak in plaats van grote interface gebruik meerdere kleinere interfaces.
    Deze interfaces zijn verantwoordelijk voor meer specifike usecases inplaats van generaliseerde usecases

    \item \textbf{Dependency inversion} 
    Een class of module zou niet moeten afhangen van implementaties maar van abstracties.
    Hier door verminder je de koppeling van de modules/classes, en verhoog je de code onderhoudbaarheid.
    Om aan dit principe te voldoen wordt er vaak gebruik gemaakt van dependency injection.
\end{enumerate}

\whitespace
Om de verschillende aspecten van SOLID te implementeren is er gekozen om gebruik te maken van handlers.
Hierbij heeft elke handler één taak hier bij kan je denken aan het ophalen of updaten van data.
Daarnaast wordt er ook gebruik gemaakt van een repository pattern om met de database te communiseren.
Er is gekozen om gebruik te maken van een repository pattern zodat er geen afhankelijkheid is van de database.
Een sequesnce diagram van deze architectuur is te zien in figuur \ref{fig:SequenceDiagramHandlerStructure}

% Er is gekozen om een \qw{Hanlder architectuur} te gebruiken.
% Deze architectuur stijl is gebruikt bij het Poiesz team met veel success.
% De structuur bestaat uit een controller die de request afhandeld.
% Vervolgens wordt er door een ketting van handlers de data verwerkt.
% En als laatste is er een access layer gemaakt die de database operaties afhandeld.
% in figuur \ref{fig:SequenceDiagramHandlerStructure} is een voorbeeld te zien van de dataflow van het systeem.

\begin{graphic}
    \captionsetup{type=figure}
    \caption{Sequence diagram Handler structuur}
    \includegraphics[scale=0.4]{SequenceDiagramArchitectureStructure.png}
    \label{fig:SequenceDiagramHandlerStructure}
\end{graphic}

\whitespace
De verschillende handlres zorgen er voor dat het single-responsibility na gevolgt wordt.
Omdat elke handler maar een taak heeft.
Verder worden er kleine interfaces gebruikt om aan het Interface segregation principle te volgen.
En als laatste worden de verschillende dependencies geinjecteerd (dependency injection pattern) om harde koppeling te voorkomen.
% Daarnaast wordt het open-closed principle gevolgt omdat handlers extend worden

% De handlers hebben maar 1 taak hier bij kun je denken aan de data ophalen of verwijderen uit de database.
% Omdat ze maar een taak hebben zijn ze makkelijk te begrijpen en te onderhouden.
% Verder wordt er gebruik gemaakt van dependency injection om de handlers minder coppeld te maken.



\newpage
\section{Process View}
dit wordt de process view

\section{Development view}
De development view is gefocust op het in beeld brengen van de organisatie van software modules en de software development omgeving \parencite{4p1Model}.
Om dit in beeld te brengen is er gebruikgemaakt van een component diagram (zie figuur \ref{fig:ComponentDiagram}).

% \todo[inline]{Ik ben het meest onzeker over dit diagram. Want ik heb een gevoel dat ik het veel te erg gegeneraliseerd heb.}

\whitespace[2]
\begin{graphic}
    \captionsetup{type=figure}
    \caption{Deployment diagram van het afstudeer product}
    \includegraphics[scale=0.4]{ComponentDiagram.png}
    \label{fig:ComponentDiagram}
\end{graphic}

\todo[inline]{Verbeter tekst development omgeving is onduidelijk}

\section{Physical view}
De physical view wordt gebruikt om aan te geven waar de software op draait (phisiek).
Om dit inbeeld te brengen is er gebruik gemaakt van een deployment diagram (zie figuur \ref{fig:DeploymentDiagram}).

\whitespace
\begin{graphic}
    \captionsetup{type=figure}
    \caption{Sequence diagram ItemValue}
    \includegraphics[scale=0.3]{SequenceDiagramItemValueWithChildren.png}
    \label{fig:DeploymentDiagram}
\end{graphic}


