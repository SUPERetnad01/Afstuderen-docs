\section{Database Keuze}
Omdat dit een nieuw datamodel is, is er ook nagedacht over welke database gebruikt zal worden.
De meeste gebruikte databases binnen Snakeware zijn SQL-databases \parencite{SQL}.
De specifieke database die het huidige CMS gebruikt is Microsoft SQL Server \parencite{MSQLServer}.
Voor de afstudeeropdracht is er gekozen om van dit pad af te stappen en een NoSQL \parencite{NoSQL} database te gebruiken.

\whitespace
Er is voor een NoSQL database oplossing gekozen omdat het belangrijk is dat het systeem zo flexibel mogelijk blijft.
Omdat het datamodel bij definitie semigestructureerde opslaat zorgt een SQL database voor extra overhead.
Verder geeft een NoSQL-database meer vrijheid in het uitbreiden van het systeem.

\whitespace
Voor de specifieke database keuzes is er gekeken naar een paar kwantitatieve punten die gebruikt worden om verschillende tegenover elkaar te zetten.
Omdat er gebruik gemaakt gaat worden van een NoSQL database zijn de volgende opties in consideratie genomen.
De vergelijkingen van de databases zijn te zien in tabel \ref{Tabel:DatabaseProviders}.

\whitespace
Een van de criteria die gebruikt wordt is ACID \parencite{ACID} wat staat voor atomicity, consistency, isolation en durability.
\begin{enumerate}
	\item{\textbf{Atomicity:} Alle veranderingen worden uit gevoerd als een operatie.
	      Als een van de veranderingen faalt alle veranderingen}
	\item{\textbf{Consistency:} De data in de database is consistent aan het begin van de transactie en aan het einde.}
	\item{\textbf{Isolation:} De stand van een transactie heeft geen involed op andere gelijktijdige transacties }
	\item{\textbf{Durability:} Als de transactie succesvol is afgerond zijn permanent en gaan niet verloren, zelfs niet bij systeemfouten  }
\end{enumerate}

In essentie biedt ACID een set van eigenschappen die ervoor zorgen dat transacties op een betrouwbare en consistente manier worden uitgevoerd in een databasesysteem, wat cruciaal is voor gegevensintegriteit en betrouwbaarheid.


\whitespace[2]
\begin{graphic}
	\captionsetup{type=table}
	\caption{Database providers vergelijking}
	\begin{tabular}{ |p{6.2em}||p{2.2cm}|p{2.2cm}|p{2.2cm}|p{2.2cm}|p{2.4cm}| }
		\hline
		\multicolumn{6}{|c|}{Database providers}                                                                                                             \\
		\hline
		Criteria       & MongoDB                     & SurrealDB   & Redis                       & CouchBase                   & Cassandra                   \\
		\hline
		Datamodel      & Document                    & Document    & key-value                   & Document                    & Wide column                 \\
		\hline
		Licentiekosten & Nee                         & Nee         & Nee                         & Ja                          & Nee                         \\
		\hline
		C\# SDK        & Ja                          & Ja          & Ja                          & Ja                          & Ja                          \\
		\hline
		Schaalbaarheid & Horizontaal\slash verticaal & Horizontaal & Horizontaal\slash verticaal & Horizontaal\slash verticaal & Horizontaal\slash verticaal \\
		\hline
		ACID           & Ja                          & Ja          & Nee                         & Nee                         & Nee                         \\
		\hline
	\end{tabular}
	\label{Tabel:DatabaseProviders}
\end{graphic}

\whitespace
Voor de specifieke database is er gekozen voor MongoDB \parencite{MongoDB}.
Deze keuze is gemaakt op basis van de verschillende criteria die gesteld zijn in tabel \ref{Tabel:DatabaseProviders}
MongoDB is een open-source document database, waarbij de data wordt opgeslagen in een json achtige format (Bson \parencite{Bson}).
Verder is MongoDB de meest gebruikte NoSQL database waardoor er veel informatie over te vinden is.
