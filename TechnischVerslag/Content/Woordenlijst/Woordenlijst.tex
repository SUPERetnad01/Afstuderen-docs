\makenoidxglossaries

\newglossaryentry{CMS}
{
	name={Contentmanagementsysteem},
	description={Een contentmanagementsysteem is een softwaretoepassing,
			meestal een webapplicatie, die het mogelijk maakt dat mensen eenvoudig, zonder veel technische kennis,
			documenten en gegevens op internet kunnen publiceren (contentmanagement).
			Als afkorting wordt ook wel CMS gebruikt},
	first={contentmanagementsysteem (CMS)},
	text={CMS}
}
\newglossaryentry{GUI}
{
	name={Graphical user interface},
	description={Een graphical user interface (GUI), is een manier van interacteren met een computer waarbij grafische beelden, widgets en tekst gebruikt worden},
	first={graphical user interfaces (GUI)},
	text={GUI}
}
\newglossaryentry{SDLC}
{
	name={Software development life cycle},
	description={De software development life cycle (SDLC) is een procesmatige manier van werken met als doel goede kwaliteit software te produceren met lage kosten in een korte tijd.
			De SDLC bestaat uit 5 fases: \textit{Requirements analysis}, \textit{Design}, \textit{Implementation}, \textit{Testing}, en \textit{Evolution} \Parencite{SDLC}},
	first={software development life cycle (SDLC)},
	text={SDLC}
}
\newglossaryentry{SEO}{
	name={Search engine optimization},
	description = {Search Engine Optimisation (SEO), zijn alle processen en verbeteringen die als doel hebben een website hoger in Google te laten verschijnen},
	first = {search engine optimization (SEO)},
	text = {SEO}
}
\newglossaryentry{UserJourneys}{
	name={User journeys},
	description = {User journeys zijn de verschillende acties die een gebruiker moet uit voeren (Meestal via een interface) om een eind resultaat te bereiken},
	first = {user journeys},
	text = {user journeys}
}
\newglossaryentry{XSL}{
	name={eXtensible Stylesheet Language (XSL)},
	description = {XSL is een XML gebaseerde opmaaktaal die voornamelijk wordt gebruikt voor het transformeren en opmaken van XML},
	first = {extensible stylesheet language (XSL)},
	text = {XSL}
}
\newglossaryentry{Stored procedures}{
	name={Stored procedures},
	description = {Een stored procedure is een voorbereide database query's die gebruikt kunnen worden als functies in andere SQL queries of als alleenstaande query's},
	first = {stored procedures},
	text = {stored procedures}
}

\newglossaryentry{Eindgebruiker}{
	name={Eindgebruiker},
    description = {Een eindgebruiker is de bezoeker van een website die ounderhouden wordt door een \gls{Beheerder}.
    Het zijn de klanten van de beheerder, een voorbeeld hiervan zijn de klanten van Poiesz supermarkten},
	first = {eindgebruiker},
	text = {eindgebruiker}
}

\newglossaryentry{Beheerder}{
	name={beheeder},
	description = {Een beheerder is een klant van Snakeware die een website onderhoud via het CMS.
    Een voorbeeld hiervan is de klant Poisz},
	first = {beheerder},
	text = {beheerder}
}
