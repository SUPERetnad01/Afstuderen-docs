\section{Aanleiding}
Het huidige platform is 21 jaar oud en er is veel functionaliteit in de loop der jaren aan toegevoegd.
Omdat Snakeware Cloud een oud platform is zijn er veel technieken en best practices gebruikt die nu niet meer als optimaal worden beschouwd.
Deze technieken waren erg geïntegreerd in Snakeware Cloud en er is in het verleden gekozen om niet de code te herschrijven om het aan de huidige standaarden te voldoen van andere projecten.
Een voorbeeld hiervan is het plaatsen van afkortingen voor de tabelnaam als prefix bij elke kolom, of gigantische C\# \Parencite{CSharp} files van 10 000 regels met verschillende functies.
Deze functies houden zich niet aan de \textit{Single Responsibility Principle} van de SOLID ontwerpmethode \Parencite{SOLID} wat het moeilijk maakt om het huidige \gls{CMS} te onderhouden.

\whitespace
Ook zijn er technieken toegepast die nu niet meer relevant zijn.
Een voorbeeld hiervan is dat het \gls{CMS} gebruikmaakt van Javascript \Parencite{JavaScript} en toen ze er mee begonnen bestonden Javascript classes \Parencite{JavascriptClasses} nog niet, dus hebben ze die zelf geïmplementeerd.
Deze oudere technieken en standaarden zorgen ervoor dat het meer tijd kost om het CMS te onderhouden vanwege de extra code.
Dit zorgt ervoor dat het meer tijd en geld kost om het Snakeware Cloud uit te breiden.
% \todo[inline]{ik wou graag bloat gebruiken inplaats van extra code maar volgens kan dat niet suggesities}
% Deze oudere technieken zorgen voor veel overhead en zorgt ervoor dat het veel tijd en geld kost om het \gls{CMS} uit te breiden.

\whitespace[2]
Een van de voornaamste uitdaging met Snakeware Cloud betreft de verouderde datastructuur van de applicatie.
Deze veroudering is het gevolg van een initïele ontwikkeling waarbij onvoldoende rekening werd gehouden met toekomstige functionaliteitsuitbreidingen in het systeem.
Als gevolg daarvan is de onderliggende datastructuur niet aangepast, maar zijn er elementen aan toegevoegd.
Dit heeft geresulteerd in database query's van duizenden regels en complexe relaties tussen tabellen in de database.
Dit huidige scenario bemoeilijkt aanzienlijk het toevoegen van nieuwe functionaliteiten, wat resulteert in aanzienlijke tijd en kosten investeringen.

\whitespace
Hierom wil Snakeware een nieuw systeem met een nieuwe datastructuur.
% Hierom wil Snakeware dat er een nieuwe datastructuur komt met de daar bij behoorende \gls{CMS}-API.
Door het gebruiken van een nieuwe softwarearchitectuur zouden er velen problemen opgelost kunnen worden die nu voor komen.
Omdat er een nieuwe datastructuur moet komen en de logica van het oude systeem nauw verbonden is met de datastructuur is het niet mogelijk om de oude code opnieuw te gebruiken.
