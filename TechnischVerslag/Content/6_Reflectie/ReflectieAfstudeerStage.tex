\section{Reflectie op afstudeerstage}
\textbf{Situatie}

\whitespace
Op 18 september 2023 ben ik begonnen bij Snakeware voor mijn afstudeerstage.
Deze afstudeerstage stage werd gedaan voor de opleiding HBO-ICT van NHL Stenden.
De afstudeerstage duurde tot 12 april.

\whitespace
\textbf{Taak}

\whitespace
Mijn taak tijdens de afstudeerstage was om een proof of concept \gls{CMS} te maken.
Dit is gedaan omdat het oude CMS verouderd is en er nieuwe mogelijkheden zijn met de bestaande technieken.
De applicatie is gerealiseerd door de \gls{SDLC} te doorlopen.

\whitespace
\textbf{Actie}
De afstudeerstage begon met het opstellen van het plan van aanpak om de grenzen van het project duidelijk te krijgen.
Daarna is de eerste stap van de \gls{SDLC} doorlopen door een requirement analyse te maken als het onderzoek voor de afstudeerstage.
Vervolgens zijn er ontwerpen gemaakt aan de hand van het 4 + 1 view model om de \textit{Designfase} van de \gls{SDLC} te doorlopen. 
Na dat het ontwerp gemaakt was is er begonnen aan het realiseren en testen van de applicatie.
De resultaten hiervan staan gedocumenteerd in het technischverslag.

\whitespace
\textbf{Resultaat}

\whitespace
Ik heb een product opgeleverd waar ik zelf zeer tevreden over ben.
Naar mijn mening is er veel potentie in het datamodel en de softwarearchitectuur die ik ontwikkeld heb.
Ook ben ik zeer trots op de verschillende documenten die ik heb opgeleverd.
Dit is niet mijn sterkste punt en ik ben zeer blij met het opgestelde document.
Ik ben er van overtuigd dat dit verslag aantoont dat ik mijn verschillende competenties heb behaald.

\whitespace
\textbf{Reflectie}
Ik had veel twijfels aan het begin van mijn afstuderen.
Dit is omdat dit niet de eerste keer is dat ik probeer afstuderen bij HBO-ICT NHL Stenden.
Deze twijfels waren meer tot betrekking van het maken van de verschillende documenten dan rond het realiseren van het product.
Ik heb veel geleerd van de verschillende documenten die ik heb op moeten leveren.

\whitespace
Tijdens het realiseren van het product heb ik ook veel feedback gekregen van mijn bedrijfsbegeleider.
Hier heb ik veel van geleerd tot betrekking van code kwaliteit en softwarearchitectuur.
