\section{Originele eisen en wensen}
In deze sectie wordt er besproken welke requirements niet mee genomen zijn voor het eindresultaat.
In sectie \ref{section:Eindproduct} wordt het gerealiseerde eindproduct besproken.
Er is besloten om 2 requirements niet te realiseren voor het eindproduct.
Deze requirements zijn: 

\whitespace
\textbf{KB-FR6, Authenticatie en identificeren}

\whitespace
Het authenticeren en identificeren van de \gls{Beheerder} is een requirement wat besloten is om te laten vallen voor het proof of concept.
De requirement was origineel verkeerd ingeschat aan hoeveel waarde het zou opleveren aan het project.
Samen met de product owner is er toen besloten om de realisatie van deze requirement een lagere prioriteit te geven.
Helaas was er geen tijd over om deze functionaliteit te implementeren.

\whitespace
\textbf{KB-FR10, Formulieren}

\whitespace
Formulieren zijn een belangrijk onderdeel voor verschillende webapplicaties.
Verder is het een belangrijk deel van het huidige \gls{CMS}.
Tijdens het ontwerpen van het datamodel kwam al naar voren dat formulieren implementeren veel werk zou kosten.
Hierom is er besloten dat formulieren niet tijdens de afstudeerperiode te realiseren.
