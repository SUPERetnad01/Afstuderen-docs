\section{Orginele eisen en wensen}
In deze sectie wordt er nagegaan welke eisen en wensen niet behaald zijn.
In sectie \ref{section:Eindproduct} wordt het gerealiseerde eindproduct besproken.
De meeste requirements zijn behaald, met uitzondering van de volgende requirements:

\whitespace
\textbf{KB-FR6, Authenticatie en indentificeren}

\whitespace
Het authenticeren en indentificeren van de \gls{Beheerder} is een requirement wat besloten is om te laten vallen voor het proof of concept.
De requirement was origneel verkeerd ingeschat aan hoeveel waarde het zou oploveren aan het project.
Samen met de product owner is er toen besloten om de realisatie van deze requirement minder te prioritarizeren.
Hellaas was er geen tijd over omdeze functionaliteit te implementeren.

\whitespace
\textbf{KB-FR10, Formulieren}

\whitespace
Formulieren zijn een belangrijk onderdeel voor verschillende webapplicaties.
Verder is het een belangrijk deel van het huidige \gls{CMS}.
Tijdens het ontwerpen van het datamodel kwam al naar voren dat formulieren implementeren veel werk zou kosten.
Hierom is er besloten dat formulieren niet tijdens het afstudeerperiode te realiseren.
