\section{Codestandaarden}
Tijdens het ontwikkelen van de afstudeeropdracht is er gecontroleerd op de code standaarden die gebruikt zijn.
Deze controles werden tijden de code reviews en stagevoortgang besproken (Zie sectie \ref{section:CodereviewsEnStageVoortgang}).

\whitespace
Voorbeelden van deze standaarden zijn het gebruik maken van expliciete types en variabele benaming conventies.
Deze conventies zijn te vinden in de bijlage (BIJLAGE TOEVOEGEN).
Verder wordt er ook gecontroleerd op het gebruik van de verschillende SOLID principes (zie sectie \ref{subsecion:SoftwareArhitectuur}.
%
% - SOLID gebruikt en gehanteerd
% - Explicit types
% - Geen naked ifs
% - Variable naming conventies.
%
%
% Omdat Snakeware niet een vaste guideline heeft voor code standaarden is er zelf een opgesteld die gevolgt is tijdens de afstudeeropdracht.
% Deze standaarden staan vermeld in de bijlage \textbf{(bron)}.
\todo[inline]{Toevoegen van code conventies bijlage}
