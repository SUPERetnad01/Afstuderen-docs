\section{Proces en bewaking}
Tijdens de afstudeerperiode is er gewerkt met een Agile \parencite{Agile} methode genaamd Scrum \Parencite{Scrum}.
Er is gewerkt met sprints van 2 weken waarbij de requirements opgedeeld werden in kleine stukjes.
Dit is gedaan om de taken behapbaar te maken en om snel veranderingen te kunnen aanbrengen.
Elke week werd er met de afstudeerbegeleider de progressie gesproken van de week waarbij mogelijk bijgestuurd werd.
De verschillende sprints werden bijgehouden in een notitie programma \textbf{Obsidian} hierbij is elke sprint opgedeeld in een aparte note.
Vervolgens is er gebruikgemaakt van een scrumboard om de progressie bij te houden dit werd ook gedaan in Obsidian.
In figuur \ref{fig:VoorbeeldSprint} is te zien hoe zo'n sprint ingepland werd.

\whitespace[2]
\begin{graphic}
	\captionsetup{type=figure}
	\caption{Sprint 4 van het realisatie proces}
	\includegraphics[scale=0.5]{SprintVoorbeeld.png}
	\label{fig:VoorbeeldSprint}
\end{graphic}
